%=============================Preamble=============================%
\documentclass[10pt,a4paper]{article}
\usepackage[T1]{fontenc}
\usepackage[utf8]{inputenc}
\usepackage[brazil]{babel}
%\usepackage[math]{anttor} % fonte um pouco mais estilizada
\everymath{\displaystyle}
\usepackage{import}
%\usepackage{parskip}
%=========================Packages==================================%
\usepackage{textcomp}
\usepackage{color,lscape, amsmath, hyperref, booktabs, latexsym, multicol, gensymb, lmodern, natbib, tikz, tkz-euclide, amssymb, enumitem, fancyhdr, lipsum, siunitx, setspace}
\usepackage{graphicx}

\newenvironment{Figure}
  {\par\medskip\noindent\minipage{\linewidth}}
  {\endminipage\par\medskip}

% configurações das questões, bem como: pontuação e estrutura.

\usepackage{tasks} % cria lista curta
\usepackage{exsheets} % cria questoes
\SetupExSheets[points]{name=ponto/s,number-format=\color{blue}} % define as configurações de pontuação das questões, e a cor da pontuação.

\DeclareInstance{exsheets-heading}{fancy-wp}{default}{
toc-reversed = true ,
indent-first = true ,
vscale = 2 ,
pre-code = \rule{\linewidth}{1pt} ,
post-code = \rule{\linewidth}{1pt} ,
title-format = \large\scshape\color{rgb:red,0.65;green,0.04;blue,0.07} ,
number-format = \large\bfseries\color{rgb:red,0.02;green,0.04;blue,0.48} ,
points-format = \itshape ,
points-pre-code = ( ,
points-post-code = ) ,
join =
{
number[r,B]title[l,B](.333em,0pt) ;
number[r,B]points[l,B](.333em,0pt)
} ,
attach = { main[hc,vc]number[hc,vc](0pt,0pt) }
}

%\SetupExSheets{headings=fancy-wp} % estilo diferente para o topo do enunciado com o nome " Exercício
%===========================Margins==============================%
\usepackage[top=8mm, bottom=20mm, left=8mm, right=8mm]{geometry}

%======================Cabeçalho e Rodapé========================%
\pagestyle{fancy}
\lfoot{\notaesquerda}
\cfoot{\thepage}
\rfoot{\notadireita}
%\lhead{HELLO}
%\chead{HELLO}
%\rhead{\textbf{The performance of new graduates}}
%\renewcommand{\headrulewidth}{0.4pt} %linha horizontal no topo da pagina
\renewcommand{\footrulewidth}{0.4pt} %linha horizontal no pé da pagina

\setlength\parindent{0pt}
\setlength\parskip{1.5ex}
\setlength\parsep{1.5\parskip}
%\thispagestyle{empty}%\bigskip %Rodapé na primeira pagina

\graphicspath{{figures/}} %informa a pasta em que as imagens estão
\usepackage{capt-of}%%To get the caption
\usepackage{amsmath}%
%=======================informações da atividade===============================%
\newcommand{\atv}{Atividade Complementar -- Funções Quadráticas}
\newcommand{\preceptor}{Professor: Fernando Jorge}
\newcommand{\turma}{1º serie}
\newcommand{\bolsistas}{\atv \\ \preceptor}

%=====================informações de rodapé=================%
\newcommand{\notaesquerda}{Funções Quadráticas
}
\newcommand{\notadireita}{Escola Estadual Professor Lima Castro}


\begin{document}

{\sf
  \begin{center}
     \textbf{\bolsistas
     }
  \end{center}
}\bigskip


\vspace{2mm}
\setlength{\marginparwidth}{5cm}
\small \noindent \textbf{Nome:}\hspace{0.3cm}\hrulefill \hrulefill
\hrulefill \hspace{0.1cm} 
\textbf{Número:}\hspace{0.1cm}\rule{1cm}{.1mm}


%\begin{center}
%\textsc{\Large Exercícios}    %Titulo do topo, antes de iniciar as questões
%\end{center}

\begin{multicols}{2}

	\setlength\columnseprule{0.6pt} % linha vertical entre as colunas
	%\newpage %% ou \clearpage ou %% \pagebreak %% força uma quebra de pagina. caso os exercicios ocupem apenas metade de uma pagina.

	\begin{center}
		\textbf{Fórmula de Bháskara}

		\begin{equation*}
			\Delta &= b^2 - 4 \cdot a \cdot c \\
			x &= \dfrac{-b \pm \sqrt{\Delta}}{2 \cdot a}
		\end{equation*}
	\end{center}

	\begin{question}[type=exam]
		Dada a função quadrática $f(x) = x^2 + 3x + 4$, determine:

		\begin{tasks}
			\task $f(0)$
			\task $f(3)$
			\task $f(10)$
			\task $f(-4)$
			\task $f(-2)$
		\end{tasks}
	\end{question}

	\begin{question}[type=exam]
		Dada a função quadrática $f(x) = 2x^2 - 3x + 1$, determine:

		\begin{tasks}
			\task $f(0)$
			\task $f(3)$
			\task $f(10)$
			\task $f(-4)$
			\task $f(-2)$
		\end{tasks}
	\end{question}

	\begin{question}[type=exam]
		Dada a função quadrática $f(x) = x^2 - x + 3$, determine:

		\begin{tasks}
			\task $f(0)$
			\task $f(3)$
			\task $f(10)$
			\task $f(-4)$
			\task $f(-2)$
		\end{tasks}
	\end{question}

	\begin{question}[type=exam]
		Dada a função quadrática $f(x) = -x^2 + 2x + 5$, determine:

		\begin{tasks}
			\task $f(0)$
			\task $f(3)$
			\task $f(10)$
			\task $f(-4)$
			\task $f(-2)$
		\end{tasks}
	\end{question}

	\begin{question}[type=exam]
		Determine as raízes das funções quadráticas, caso existam, abaixo:

		\begin{tasks}
			\task $y=x^2 - 1$
			\task $y=x^2 + 3x + 2$
			\task $y = x^2 + x - 2$
			\task $y=x^2 - 6x +9$
			\task $y=x^2 -4x + 3$
			\task $y=x^2 -x -2$
			\task $y = x^2 -2x -3$
		\end{tasks}

	\end{question}

	\begin{center}
		\textbf{Gabarito}
	\end{center}

	\textbf{Questão 1.}

	\begin{tasks}(5)
		\task 4
		\task 22
		\task 134
		\task 8
		\task 2
	\end{tasks}

	\textbf{Questão 2.}

	\begin{tasks}(5)
		\task 1
		\task 10
		\task 171
		\task 45
		\task 15
	\end{tasks}


	\textbf{Questão 3.}

	\begin{tasks}(5)
		\task 3
		\task 9
		\task 93
		\task 23
		\task 9
	\end{tasks}

	\textbf{Questão 4.}

	\begin{tasks}(5)
		\task 5
		\task 2
		\task -75
		\task -19
		\task -3
	\end{tasks}

	\textbf{Questão 5.}

	\begin{tasks}(2)
		\task $S = \{-1, 1\}$
		\task $S = \{-1, -2\}$
		\task $S = \{1, -2\}$
		\task $S = \{3\}$
		\task $S = \{1, 3\}$
		\task $S = \{-1, 3\}$
	\end{tasks}

	\clearpage
\end{multicols}

%\import{questions/}{q7} % exemplo para mostrar que pode colocar questões fora das colunas e mesclar os estilos. Recomendado adicionar questões que incluem imagens, ao final e fora das colunas.






\end{document}
