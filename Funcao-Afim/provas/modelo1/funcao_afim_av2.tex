%%Template made by Uday Khankhoje for examinations using the exam template
%%Refer to the documentation http://www-math.mit.edu/~psh/exam/examdoc.pdf
%%for lot more bells and whistles to the standard template shown below
\documentclass[a4paper,11pt,addpoints]{exam}
\usepackage[left=1.5cm,right=1.5cm,top=1.5cm,bottom=2cm]{geometry}
%\usepackage{mathrsfs}
\usepackage{graphicx,color}
\usepackage[x11names]{xcolor}
\usepackage{venndiagram}
\usepackage{tikz}
\usepackage{tkz-fct}
\usepackage{tkz-euclide}
\usepackage{epic,eepic}
%\usepackage{mathpazo}
\usepackage{url}
\usepackage{tasks} % cria lista curta
\usepackage{multicol}
\usepackage{pgfplots}
\usepackage{amsmath, amsthm, amssymb}
\pointsinmargin
\boxedpoints
\renewcommand*\half{.5}
\usepackage{setspace}
\DeclareMathOperator{\vecc}{vec}

\pgfplotsset{compat=1.18, width=7cm}
%\renewcommand{\vec}[1]{\ensuremath{\mathbf{#1}}}

\global\vbadness=1616

\begin{document}
\noindent
%%PART 1 of header
\begin{center}
	\vspace*{-3em}
	\def\arraystretch{2.0}
	\begin{tabular}{|p{0.7\linewidth}|p{0.2\linewidth}|}
		\hline
		\textbf{Avaliação de Matemática - Terceiro Bimestre}                                                           & Pontos Obtidos $\downarrow$ \\
		\hline
		Data:\hspace{3cm}  Total de questões \textbf{\numquestions} \hspace{1cm} Total de pontos: \textbf{\numpoints} &                             \\
		\hline
		\multicolumn{2}{|l|}{Tuma: \hspace{0.3\linewidth} Nome: \hspace{0.3\linewidth} Duração: 1 hr}                                               \\
		\hline
	\end{tabular}
\end{center}
%%PART 2 of header, if you have too many questions, this may be a problem
%%if so, use \multirowgradetable{n}[questions], where n is the number of rows you want
%%or, switch to \gradetable[h][pages] instead,
\begin{center}
	\gradetable[h][questions]
\end{center}
%%PART 3 of header
\textbf{Instruções
	\begin{enumerate}
		\item Necessário todos os cálculos.
        \item Questões de múltipla escolha sem cálculos serão desconsideradas.
	\end{enumerate}
}
%%toggle comment on next line to show/hide the answers
% \printanswers
%%Now the actual paper!
\begin{questions}

    \question[1]

    Curiosamente, após uma madrugada chuvosa, observou-se que no período das 9
    às 18 horas a variação da temperatura em uma cidade decresceu linearmente.
    Se, nesse dia, às 9 horas os termômetros marcavam $32^\circ C$ e, às 18
    horas, $20^\circ C$, então às 12 horas a temperatura era de:

    \begin{tasks}(5)
        \task $25^\circ C$
        \task $26.5^\circ C$
        \task $27^\circ C$
        \task $27.5^\circ C$
        \task $28^\circ C$
    \end{tasks}

    \question[1]

    Dada a função afim $f(x) = -4x -5$, determine:

    \begin{tasks}(2)
        \task $f(0)$
        \task $f(-3)$
        \task $f(4)$
        \task $f(2)$
    \end{tasks}

    \question[1]

    Determine os zeros ou raízes das funções:

    \begin{tasks}(2)
        \task $f(x) = -5x - 30$
        \task $f(x) = 4x - 16$
        \task $f(x) = 2x + 4$
        \task $f(x) = 2x$
    \end{tasks}

    \question[1]

    Um botânico mede o crescimento de uma planta, em centímetros, todos os dias.
    Ligando os pontos, colocados por ele, num gráfico, resulta o gráfico abaixo.

    \begin{center}
        \begin{tikzpicture}
            \begin{axis}[
                xmin=0,
                xmax=15,
                ymin=0,
                ymax=3,
                axis lines=center,
                xlabel=t (dias),
                ylabel=h (cm),
                xtick={0,5,10},
                ytick={1,2}
                ]
                \addplot[color=blue, domain=0:12]{x/5};
                \addplot[only marks] table {
                        5 1
                        10 2
                    };
                \addplot[dashed, color=black]{1};
                \addplot[dashed, domain=0:10, color=black]{2};
                \addplot[dashed] coordinates{(10,0)(10,2)};
                \addplot[dashed] coordinates{(5,0)(5,1)};
            \end{axis}
        \end{tikzpicture}
    \end{center}

    Se mantida sempre essa relação entre tempo e altura, a planta terá no
    quadragésimo dia, uma altura em centímetros igual a:

    \begin{tasks}(5)
        \task 6
        \task 8
        \task 12
        \task 15
        \task 30
    \end{tasks}

    \question[2]

    Faça o esboço do gráfico das funções abaixo:

    \begin{tasks}
        \task $f(x) = x + 2$
        \task $f(x) = 2x + 1$
        \task $f(x) = -x + 4$
        \task $f(x) = -3x + 2$
        \task $f(x) = 5x$
    \end{tasks}

\end{questions}
\end{document}
