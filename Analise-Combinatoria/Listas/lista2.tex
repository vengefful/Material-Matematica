%=============================Preamble=============================%
\documentclass[10pt,a4paper]{article}
\usepackage[T1]{fontenc}
\usepackage[utf8]{inputenc}
\usepackage[brazil]{babel}
%\usepackage[math]{anttor} % fonte um pouco mais estilizada
\everymath{\displaystyle}
\usepackage{import}
%\usepackage{parskip}
%=========================Packages==================================%
\usepackage{textcomp}
\usepackage{color,lscape, amsmath, hyperref, booktabs, latexsym, multicol, gensymb, lmodern, natbib, tikz, tkz-euclide, amssymb, enumitem, fancyhdr, lipsum, siunitx, setspace}
\usepackage{graphicx}

\newenvironment{Figure}
  {\par\medskip\noindent\minipage{\linewidth}}
  {\endminipage\par\medskip}

% configurações das questões, bem como: pontuação e estrutura.

\usepackage{tasks} % cria lista curta
\usepackage{exsheets} % cria questoes
\SetupExSheets[points]{name=ponto/s,number-format=\color{blue}} % define as configurações de pontuação das questões, e a cor da pontuação.

\DeclareInstance{exsheets-heading}{fancy-wp}{default}{
toc-reversed = true ,
indent-first = true ,
vscale = 2 ,
pre-code = \rule{\linewidth}{1pt} ,
post-code = \rule{\linewidth}{1pt} ,
title-format = \large\scshape\color{rgb:red,0.65;green,0.04;blue,0.07} ,
number-format = \large\bfseries\color{rgb:red,0.02;green,0.04;blue,0.48} ,
points-format = \itshape ,
points-pre-code = ( ,
points-post-code = ) ,
join =
{
number[r,B]title[l,B](.333em,0pt) ;
number[r,B]points[l,B](.333em,0pt)
} ,
attach = { main[hc,vc]number[hc,vc](0pt,0pt) }
}

%\SetupExSheets{headings=fancy-wp} % estilo diferente para o topo do enunciado com o nome " Exercício
%===========================Margins==============================%
\usepackage[top=8mm, bottom=20mm, left=8mm, right=8mm]{geometry}

%======================Cabeçalho e Rodapé========================%
\pagestyle{fancy}
\lfoot{\notaesquerda}
\cfoot{\thepage}
\rfoot{\notadireita}
%\lhead{HELLO}
%\chead{HELLO}
%\rhead{\textbf{The performance of new graduates}}
%\renewcommand{\headrulewidth}{0.4pt} %linha horizontal no topo da pagina
\renewcommand{\footrulewidth}{0.4pt} %linha horizontal no pé da pagina

\setlength\parindent{0pt}
\setlength\parskip{1.5ex}
\setlength\parsep{1.5\parskip}
%\thispagestyle{empty}%\bigskip %Rodapé na primeira pagina

\graphicspath{{figures/}} %informa a pasta em que as imagens estão
\usepackage{capt-of}%%To get the caption
\usepackage{amsmath}%
\usepackage{tasks}
%=======================informações da atividade===============================%
\newcommand{\atv}{Lista de Exercícios 02 -- Análise Combinatória}
\newcommand{\preceptor}{Professor: Fernando Jorge}
\newcommand{\turma}{2º serie}
\newcommand{\bolsistas}{\atv \\ \preceptor}

%=====================informações de rodapé=================%
\newcommand{\notaesquerda}{Análise Combinatória
}
\newcommand{\notadireita}{Escola Estadual Professor Lima Castro}


\begin{document}

{\sf
  \begin{center}
     \textbf{\bolsistas
     }
  \end{center}
}\bigskip


\vspace{2mm}
\setlength{\marginparwidth}{5cm}
\small \noindent \textbf{Nome:}\hspace{0.3cm}\hrulefill \hrulefill
\hrulefill \hspace{0.1cm} 
\textbf{Número:}\hspace{0.1cm}\rule{1cm}{.1mm}


%\begin{center}
%\textsc{\Large Exercícios}    %Titulo do topo, antes de iniciar as questões
%\end{center}

	\begin{multicols}{2}

	  \setlength\columnseprule{0.6pt} % linha vertical entre as colunas
	  %\newpage %% ou \clearpage ou %% \pagebreak %% força uma quebra de pagina. caso os exercicios ocupem apenas metade de uma pagina.

    \begin{question}[type=exam]
        Um ovo de brinquedo contém no seu interior duas figurinhas distintas, um
        bonequinho e um docinho. Sabe-se que na produção dessa brinquedo, há
        disponível para escolha 20 figurinhas, 10 bonequinhos e 4 docinhos, todos
        distintos. O número de maneiras que se pode compor o interior desse ovo de
        brinquedo é:
        \begin{tasks}(5)
            \task 15200
            \task 7600
            \task 3800
            \task 800
            \task 400
        \end{tasks}
    \end{question}

    \begin{question}[type=exam]
        Certo departamento de uma empresa tem como funcionários exatamente
        oito mulheres e seis homens. A empresa solicitou ao departamento que
        enviasse uma comissão formada por três mulheres e dois homens para
        participar de uma reunião. O departamento pode atender à solicitação
        de quantas maneiras diferentes.

        \begin{tasks}(5)
            \task 840
            \task 720
            \task 401
            \task 366
            \task 71
        \end{tasks}
    \end{question}

    \begin{question}[type=exam]
        De quantas maneiras diferentes podemos escolher seis pessoas, incluindo
        pelo menos duas mulheres, de um grupo composto de sete homens e mulheres?
        \begin{tasks}(5)
            \task 210
            \task 250
            \task 371
            \task 462
            \task 756
        \end{tasks}
    \end{question}

    \begin{question}[type=exam]
        Quantos são os números inteiros positivos com três digitos distintos nos quais
        o algarismo 5 aparece?
        \begin{tasks}(5)
            \task 136
            \task 200
            \task 176
            \task 194
        \end{tasks}
    \end{question}

    \begin{question}[type=exam]
        A turma de espanhol de uma escola é composta por 20 estudantes. Serão
        formados grupos de três estudantes para uma apresentação cultural. De
        quantas maneiras se podem formar esses grupos, sabendo que dois dos
        estudantes não podem pertencer a um mesmo grupo?
        \begin{tasks}(5)
            \task 6840
            \task 6732
            \task 4896
            \task 1836
            \task 1122
        \end{tasks}
    \end{question}

    \begin{question}[type=exam]
        O número de anagramas que se pode formar com as letras da palavra
        ARRANJO é igual a:
        \begin{tasks}(5)
            \task 21
            \task 42
            \task 5040
            \task 2520
            \task 1260
        \end{tasks}
    \end{question}

    \begin{question}[type=exam]
        Existem 6 caminhos diferentes ligando as escolas E1 e E2 e 4 caminhos
        diferentes ligando as escolas E2 e E3. De quantas maneiras é possível
        ir da escola E1 para a escola E3, passando por E2?

        \begin{taks}(5)
            \task 15
            \task 10
            \task 12
            \task 24
            \task 360
        \end{taks}
    \end{question}

    \begin{question}[type=exam]
        O grêmio estudantil de uma escola é composto por 6 alunos e 8 alunas.
        Na última reunião do grêmio, decidiu-se formar umna comissão de 3
        rapazes e 5 moças para a organização das olimpíadas do colégio.

        De quantos modos diferentes pode-se formar essa comissão?

        \begin{tasks}(5)
            \taks 1120
            \task 2240
            \task 6720
            \task 100800
            \task 806400
        \end{tasks}
    \end{question}

    \begin{question}[type=exam]
        Calcule o número de anagramas da palavra TEORIA que tenham as letras
        T e R juntas e nesta ordem.
    \end{question}

    \begin{question}[type=exam]
        Se em uma reunião com 10 pessoas, todas se cumprimentam, quantos cumprimentos
        houveram?
    \end{question}

    \begin{question}[type=exam]
        De um grupo de seis pessoas, quantas comissões com três pessoas, sendo
        um presidente, um secretário e um conselheiro, podemos formar?
    \end{question}

    \begin{question}[type=exam]
        Calcule a quantidade de números de seis algarismos que se pode criar, sendo
        que algarismos consecutivos sejam distintos.
    \end{question}

    \begin{question}[type=exam]
        Uma prova de múltipla escolha tem 10 questões, cada qual com 4 alternativas.
        De quantas maneiras diferentes um aluno pode responder toda a prova?
    \end{question}

    \begin{question}[type=exam]
        Quantos números possuem exatamente quatro algarismos?
    \end{question}

    \begin{question}[type=exam]
        Dos números de quatro algarismos, quantos são pares?
    \end{question}

    \begin{question}[type=exam]
        Dos números de quatro algarismos, quantos não tem algarismos repetidos?
    \end{question}

    \begin{question}[type=exam]
        No país da Vogal, as placas de licença de automóveis são formados por
        3 letras, seguidas de 4 algarismos, sendo as letras escolhidas apenas
        entre as vogais A, E, I, O e U, e sendo os algarismos distintos e escolhidos
        entre os algarismos de 0 a 9.

        \begin{tasks}
            \task Qual é o maior número de placas de licença de automóveis que podem
            ser formadas em tal país?
            \task Quantas dessas placas tem os algarismos formando um múltiplo de 5?
            Que porcentagem do total esse número representa?
        \end{tasks}
    \end{question}

    \begin{question}[type=exam]
        Se uma moeda é jogada para cima quatro vezes, quantas sequências diferentes de
        cara e coroa podem ser produzidas?
    \end{question}

    \begin{question}[type=exam]
        Um site de relacionamentos possui o cadastro de 150 homens e 200 mulheres
        com idade entre 18 e 25 anos. Quantos casais diferentes, nessa faixa etária,
        podem surgir a partir desse site?
    \end{question}

    \begin{question}[type=exam]
        Um \textit{motoboy} precisa entregar quatro pizzas. De quantas maneiras
        diferentes ele pode visitar os quatro clientes da pizzaria?
    \end{question}

    \begin{question}[type=exam]
        A senha de um \textit{site} de compras possui 6 caracteres, incluindo as letras
        do alfabeto (o site não distingue letras minúsculas de maiúsculas) e os algarismos de 0 a 9. Quantas senhas diferentes um cliente
        pode gerar?
    \end{question}

    \begin{question}[type=exam]
        Quinze times de basquete se enfrentam em um torneio no qual cada time
        joga contra todos os outros, em  turno e returno. Quantas partidas são
        disputadas no torneio?
    \end{question}

    \begin{question}[type=exam]
        Quantos anagramas possui a palavra SURTO?
    \end{question}

    \begin{question}[type=exam]
        Doze pessoas se candidataram ao DCE. Pelas regras eleitorais, o candidato
        mais votado é nomeado presidente do diretório, cabendo ao segundo mais
        votado o cargo de vice-presidente. Quantas diretorias distintas podem
        ser eleitas?
    \end{question}

    \begin{question}[type=exam]
        Dois prêmios iguais serão sorteados entre vinte pessoas, das quais
        doze são mulheres e oito são homens. Admitindo que uma pessoa não possa
        ganhar os dois prêmios:
        \begin{tasks}
            \task De quantas maneiras diferentes pode-se distribuir os prêmios entre as pessoas?
            \task De quantas maneiras diferentes pode-se distribuir os prêmios se
            um deve ser concedido a uma mulher e o outro a um homem
        \end{tasks}
    \end{question}

    \begin{question}[type=exam]
        Um sargento deve selecionar 5 soldados para uma missão, dentre os 12 que estão
        sob seu comando no momento. De quantas formas ele pode selecionar os soldados?
    \end{question}

    \begin{question}[type=exam]
        A diretoria de uma empresa é constituída por 7 brasileiros e 4 japoneses.
        Quantas comissões de 3 brasileiros e 3 japoneses podem ser formadas?
    \end{question}

    \begin{question}[type=exam]
        Quantos números naturais pares de três algarismos distintos existem com os
        algarismos 1, 2, 3, 4, 5, 6 e 9?
    \end{question}

    \begin{question}[type=exam]
        Com os algarismos pares, sem os repetir, quantos números naturais compreendidos
        entre 2000 e 7000 podem ser formados?
    \end{question}

    \begin{question}[type=exam]
        Quantos são os gabaritos possíveis de um teste de 10 questões de múltipla escolha,
        com cinco alternativas por questão?
    \end{question}

    \begin{question}[type=exam]
        De quantos modos 3 pessoas podem sentar-se em 5 cadeiras em fila?
    \end{question}

    \begin{question}[type=exam]
        A quantidade de números naturais de três algarismos com \textbf{pelo menos} dois
        algarismos iguais é:

        \begin{tasks}(5)
            \task 38
            \task 252
            \task 300
            \task 414
            \task 454
        \end{tasks}
    \end{question}

    \begin{question}[type=exam]
        Quantos são os números de 5 algarismos nos quais o algarismo ``2'' aparece?
    \end{question}

    \begin{question}[type=exam]
        Resolver a equação $\dfrac{(p+2)!}{p!} = 72$
    \end{question}

    \begin{question}[type=exam]
        Os números dos telefones de uma cidade são constituídos por 6 digítos.
        Sabendo que o primeiro digíto nunca pode ser zero e que os números dos
        telefones passarão a ser de 7 digítos, o aumento na quantidade dos telefones
        será:

        \begin{tasks}(5)
            \task $81 \cdot 10^3$
            \task $90 \cdot 10^3$
            \task $81 \cdot 10^4$
            \task $81 \cdot 10^5$
            \task $90 \cdot 10^5$
        \end{tasks}
    \end{question}

    \begin{question}[type=exam]
        De quantas maneiras 10 clientes de um banco podem se posicionar na fila
        única dos caixas de modo que as 4 mulheres do grupo fiquem juntas?

        \begin{tasks}(5)
            \task $4! \cdot 7!$
            \task $5! \cdot 6!$
            \task $6! \cdot 6!$
            \task $10! \cdot 6!$
            \task $4!+10!$
        \end{tasks}
    \end{question}

    \begin{question}[type=exam]
        A partir de um grupo de 12 professores, quer se formar uma comissão com
        um presidente, um relator e cinco outros membros. O número de formas
        de se compor a comissão é:

        \begin{tasks}(5)
            \task 12772
            \task 13024
            \task 25940
            \task 33264
            \task 27764
        \end{tasks}
    \end{question}

    \begin{question}[type=exam]
        Uma prova de atletismo é disputada por 9 atletas, dos quais apenas 4 são
        brasileiros. Os resultados possíveis para a prova, de modo que \textbf{pelo menos}
        um brasileiro fique numa das três primeiras colocações, são em número de:

        \begin{tasks}(5)
            \task 426
            \task 444
            \task 468
            \task 480
            \task 504
        \end{tasks}
    \end{question}

    \begin{question}[type=exam]
        Você faz parte de um grupo de 12 pessoas, 5 das quais deverão ser selecionadas
        para formar um grupo de trabalho. De quantos modos você poderá fazer parte do grupo
        a ser formado?

        \begin{tasks}(5)
            \task 182
            \task 330
            \task 462
            \task 782
            \task 7920
        \end{tasks}

    \end{question}

    \begin{question}[type=exam]
        Num grupo de 10 pessoas, temos somente 2 homens. O número de comissões
        de 5 pessoas que podemos formar com 1 homem e 4 mulheres é:

        \begin{tasks}(5)
            \task 70
            \task 84
            \task 140
            \task 210
            \task 252
        \end{tasks}
    \end{question}

    \begin{question}[type=exam]
        Assinale a alternativa na qual consta a quantidade de números inteiros
        formados por três algarismos distintos, escolhidos dentre 1, 3, 5, 7 e 9,
        e que são maiores que 200 e menores que 800.

        \begin{tasks}(5)
            \task 30
            \task 36
            \task 42
            \task 48
            \task 52
        \end{tasks}
    \end{question}

    \begin{question}[type=exam]
        Dentre 6 números positivos e 6 números negativos, de quantos modos
        podemos escolher 4 números cujo produto seja positivo?

        \begin{tasks}(5)
            \task 255
            \task 960
            \task 30
            \task 625
            \task 720
        \end{tasks}
    \end{question}

    \begin{question}[type=exam]
        O número de permutações distintas possíveis com as 9 letras da palavra
        PARALELA, começando todas com a letra P, será de:

        \begin{tasks}(5)
            \task 120
            \task 720
            \task 420
            \task 24
            \task 360
        \end{tasks}
    \end{question}

    \begin{question}[type=exam]
        Usando os algarismos do conjunto $\{2, 6\}$, podemos formar quantos números
        de 4 algarismos?

        \begin{tasks}(5)
            \task 0
            \task 2
            \task 4
            \task 12
            \task 16
        \end{tasks}
    \end{question}

    \begin{question}[type=exam]
        Quantos números pares de 5 algarismos podemos escrever apenas com os
        digítos 1, 1, 2, 2 e 3, respeitadas as repetições apresentas?

        \begin{tasks}(5)
            \task 12
            \task 30
            \task 6
            \task 24
            \task 18
        \end{tasks}
    \end{question}


		\clearpage
	\end{multicols}

  %\import{questions/}{q7} % exemplo para mostrar que pode colocar questões fora das colunas e mesclar os estilos. Recomendado adicionar questões que incluem imagens, ao final e fora das colunas.






\end{document}
