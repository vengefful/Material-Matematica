%%Template made by Uday Khankhoje for examinations using the exam template
%%Refer to the documentation http://www-math.mit.edu/~psh/exam/examdoc.pdf
%%for lot more bells and whistles to the standard template shown below
\documentclass[a4paper,11pt,addpoints]{exam}
\usepackage[left=1.5cm,right=1.5cm,top=1.5cm,bottom=2cm]{geometry}
%\usepackage{mathrsfs}
\usepackage{graphicx,color}
\usepackage[x11names]{xcolor}
\usepackage{venndiagram}
\usepackage{epic,eepic}
%\usepackage{mathpazo}
\usepackage{url}
\usepackage{tasks} % cria lista curta
\usepackage{multicol}
\usepackage{amsmath, amsthm, amssymb}
\pointsinmargin
\boxedpoints
\renewcommand*\half{.5}
\usepackage{setspace}
\DeclareMathOperator{\vecc}{vec}
%\renewcommand{\vec}[1]{\ensuremath{\mathbf{#1}}}

\global\vbadness=1616

\begin{document}
\noindent
%%PART 1 of header
\begin{center}
	\vspace*{-3em}
	\def\arraystretch{2.0}
	\begin{tabular}{|p{0.7\linewidth}|p{0.2\linewidth}|}
		\hline
		\textbf{Avaliação de Matemática - Terceiro Bimestre}                                                          & Pontos Obtidos $\downarrow$ \\
		\hline
		Data:\hspace{3cm}  Total de questões \textbf{\numquestions} \hspace{1cm} Total de pontos: \textbf{\numpoints} &                             \\
		\hline
		\multicolumn{2}{|l|}{Tuma: \hspace{0.3\linewidth} Nome: \hspace{0.3\linewidth} Duração: 1 hr}                                               \\
		\hline
	\end{tabular}
\end{center}
%%PART 2 of header, if you have too many questions, this may be a problem
%%if so, use \multirowgradetable{n}[questions], where n is the number of rows you want
%%or, switch to \gradetable[h][pages] instead,
\begin{center}
	\gradetable[h][questions]
\end{center}
%%PART 3 of header
\textbf{Instruções
	\begin{enumerate}
		\item Explique todas as questões claramente.
		\item Necessário todos os cálculos.
	\end{enumerate}
}
%%toggle comment on next line to show/hide the answers
% \printanswers
%%Now the actual paper!
\begin{questions}

	\question[1]

	O número de anagramas que se pode formar com as letras da palavra
	\textbf{ARRANJO} é igual a:

	\begin{tasks}(5)
		\task 21
		\task 42
		\task 5040
		\task 2520
		\task 1260
	\end{tasks}

	\question[1]

	Se uma reunião com 10 pessoas, todas se cumprimentam, quantos cumprimentos
	houveram?

	\question[1]

	Um sargento deve selecionar 5 soldados para uma missão, dentre os 12
	que estão sob seu comando no momento. De quantas formas ele pode
	selecionar os soldados?

	\question[1]

	Com os algarismos pares sem os repetir, quantos números naturais
	compreendidos entre 2000 e 7000 podem ser formados?

	\question[1]

	De quantos modos 3 pessoas podem sentar-se em 5 cadeiras em fila?

	\question[1]

	Num grupo de 10 pessoas, temos somente 2 homens. O número de comissões de 5
	pessoas que podemos formar com 1 homem e 4 mulheres é:

	\begin{tasks}(5)
		\task 70
		\task 84
		\task 140
		\task 210
		\task 252

	\end{tasks}


\end{questions}
\end{document}
