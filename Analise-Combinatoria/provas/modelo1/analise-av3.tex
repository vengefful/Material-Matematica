%%Template made by Uday Khankhoje for examinations using the exam template
%%Refer to the documentation http://www-math.mit.edu/~psh/exam/examdoc.pdf
%%for lot more bells and whistles to the standard template shown below
\documentclass[a4paper,11pt,addpoints]{exam}
\usepackage[left=1.5cm,right=1.5cm,top=1.5cm,bottom=2cm]{geometry}
%\usepackage{mathrsfs}
\usepackage{graphicx,color}
\usepackage[x11names]{xcolor}
\usepackage{venndiagram}
\usepackage{epic,eepic}
%\usepackage{mathpazo}
\usepackage{url}
\usepackage{tasks} % cria lista curta
\usepackage{multicol}
\usepackage{amsmath, amsthm, amssymb}
\pointsinmargin
\boxedpoints
\renewcommand*\half{.5}
\usepackage{setspace}
\DeclareMathOperator{\vecc}{vec}
%\renewcommand{\vec}[1]{\ensuremath{\mathbf{#1}}}

\global\vbadness=1616

\begin{document}
\noindent
%%PART 1 of header
\begin{center}
	\vspace*{-3em}
	\def\arraystretch{2.0}
	\begin{tabular}{|p{0.7\linewidth}|p{0.2\linewidth}|}
		\hline
		\textbf{Avaliação de Matemática - Terceiro Bimestre}                                                          & Pontos Obtidos $\downarrow$ \\
		\hline
		Data:\hspace{3cm}  Total de questões \textbf{\numquestions} \hspace{1cm} Total de pontos: \textbf{\numpoints} &                             \\
		\hline
		\multicolumn{2}{|l|}{Tuma: \hspace{0.3\linewidth} Nome: \hspace{0.3\linewidth} Duração: 1 hr}                                               \\
		\hline
	\end{tabular}
\end{center}
%%PART 2 of header, if you have too many questions, this may be a problem
%%if so, use \multirowgradetable{n}[questions], where n is the number of rows you want
%%or, switch to \gradetable[h][pages] instead,
\begin{center}
	\gradetable[h][questions]
\end{center}
%%PART 3 of header
\textbf{Instruções
	\begin{enumerate}
		\item Explique todas as questões claramente.
		\item Necessário todos os cálculos.
	\end{enumerate}
}
%%toggle comment on next line to show/hide the answers
% \printanswers
%%Now the actual paper!
\begin{questions}

	\question[1]

	Quantos anagramas possui a palavra \textbf{TESTES}?

	\question[1]

	Uma classe de 20 alunos precisa escolher um líder e um ajudante para
	organizar a turma para a gincana da escola. De quantos modos a classe
	pode fazer esta escolha?

	\question[1]

	Um partido tem 5 nomes distintos para compor a chapa (prefeito e vice-prefeito)
	que vai concorrer à prefeitura municipal. De quanntos modos o partido pode
	montar a chapa?

	\question[1]

	Num ônibus, há 6 lugares vagos. Se 3 pessoas entram no ônibus, de quantas
	maneiras elas podem sentar-se no ônibus?

	\question[1]

	Certo dia, o pai de Carlos emprestou-lhe o cartão de crédito e revelou a
	senha do mesmo, porém Carlos esqueceu a senha revelada. Apenas lembrou-se
	de que os 4 dígitos da senha eram distintos e de que o segundo dígito era
	igual a 4 vezes o primeiro. Tentando chutar a senha do cartão do seu pai,
	até quantas vezes Carlos pode errar esse chute, sabendo que ele nunca
	repete um chute errado?

	\question[1]

	Quantos anagramas da palavra \textbf{EDITORA}:

	\begin{tasks}
		\task começam por A?
		\task começam por A e terminam por E?
	\end{tasks}







\end{questions}
\end{document}
