%%Template made by Uday Khankhoje for examinations using the exam template
%%Refer to the documentation http://www-math.mit.edu/~psh/exam/examdoc.pdf
%%for lot more bells and whistles to the standard template shown below
\documentclass[a4paper,11pt,addpoints]{exam}
\usepackage[left=1.5cm,right=1.5cm,top=1.5cm,bottom=2cm]{geometry}
%\usepackage{mathrsfs}
\usepackage{graphicx,color}
\usepackage[x11names]{xcolor}
\usepackage{venndiagram}
\usepackage{epic,eepic}
%\usepackage{mathpazo}
\usepackage{url}
\usepackage{tasks} % cria lista curta
\usepackage{multicol}
\usepackage{amsmath, amsthm, amssymb}
\pointsinmargin
\boxedpoints
\renewcommand*\half{.5}
\usepackage{setspace}
\DeclareMathOperator{\vecc}{vec}
%\renewcommand{\vec}[1]{\ensuremath{\mathbf{#1}}}

\global\vbadness=1616

\begin{document}
\noindent
%%PART 1 of header
\begin{center}
	\vspace*{-3em}
	\def\arraystretch{2.0}
	\begin{tabular}{|p{0.7\linewidth}|p{0.2\linewidth}|}
		\hline
		\textbf{Avaliação de Matemática - Terceiro Bimestre}                                                           & Pontos Obtidos $\downarrow$ \\
		\hline
		Data:\hspace{3cm}  Total de questões \textbf{\numquestions} \hspace{1cm} Total de pontos: \textbf{\numpoints} &                             \\
		\hline
		\multicolumn{2}{|l|}{Tuma: \hspace{0.3\linewidth} Nome: \hspace{0.3\linewidth} Duração: 1 hr}                                               \\
		\hline
	\end{tabular}
\end{center}
%%PART 2 of header, if you have too many questions, this may be a problem
%%if so, use \multirowgradetable{n}[questions], where n is the number of rows you want
%%or, switch to \gradetable[h][pages] instead,
\begin{center}
	\gradetable[h][questions]
\end{center}
%%PART 3 of header
\textbf{Instruções
	\begin{enumerate}
		\item Explique todas as questões claramente.
		\item Necessário todos os cálculos.
	\end{enumerate}
}
%%toggle comment on next line to show/hide the answers
% \printanswers
%%Now the actual paper!
\begin{questions}

	\question[1]

    Dezesseis times de basquete se enfrentam em um torneio no qual cada time joga
    contra todos os outros, em turno e returno. Quantas partidas são disputadas
    no torneio?

    \question[1]

    Uma prova de atletismo é disputada por 9 atletas, dos quais apenas 4 são
    brasileiros. Os resultados possíveis para a prova, de modo que \textbf{pelo menos} um
    brasileiro fique numa das três primeiras colocações, são em número de:

    \begin{tasks}(5)
        \task 426
        \task 444
        \task 468
        \task 480
        \task 504
    \end{tasks}

    \question[1]

    Assinale a alternativa na qual consta a quantidade de números inteiros
    formados por três algarismos distintos, escolhidos dentre 1, 3, 5, 7 e 9,
    e que são maiores que 200 e menores que 800 são:

    \begin{tasks}(5)
        \task 30
        \task 36
        \task 42
        \task 48
        \task 52
    \end{tasks}

    \question[1]

    O número de permutações distintas com as 9 letras da palavra PARALELA,
    começando todas com a letra \textbf{P}, será de:

    \begin{tasks}(5)
        \task 120
        \task 720
        \task 420
        \task 24
        \task 360
    \end{tasks}

    \question[1]

    Dois prêmios iguais serão sorteados entre vinte pessoas, das quais doze são
    mulheres e oito são homens. Admitindo que uma pessoa não possa ganhar os
    dois prêmios:

    \begin{tasks}
        \task De quantas maneiras diferentes pode-se distribuir os prêmios entre
        as pessoas?
        \task De quantas maneiras diferentes pode-se distribuir os prêmios
        se um deve ser concedido a uma mulher e o outro a um homem?
    \end{tasks}

    \question[1]

    De quantas maneiras dez clientes de um banco podem se posicionar na fila
    única dos caixas de modo que as 4 mulheres do grupo fiquem juntas?

    \begin{tasks}(5)
        \task $4! \cdot 7!$
        \task $5! \cdot 6!$
        \task $6! \cdot 6!$
        \task $10! \cdot 6!$
        \task $4! + 10!$

    \end{tasks}


\end{questions}
\end{document}
