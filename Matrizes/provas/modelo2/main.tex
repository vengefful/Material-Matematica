%%Template made by Uday Khankhoje for examinations using the exam template
%%Refer to the documentation http://www-math.mit.edu/~psh/exam/examdoc.pdf
%%for lot more bells and whistles to the standard template shown below
\documentclass[a4paper,11pt,addpoints]{exam}
\usepackage[left=1.5cm,right=1.5cm,top=1.5cm,bottom=2cm]{geometry}
%\usepackage{mathrsfs}
\usepackage{graphicx,color}
\usepackage[x11names]{xcolor}
\usepackage{venndiagram}
\usepackage{epic,eepic}
%\usepackage{mathpazo}
\usepackage{url}
\usepackage{tasks} % cria lista curta
\usepackage{multicol}
\usepackage{amsmath, amsthm, amssymb}
\pointsinmargin
\boxedpoints
\renewcommand*\half{.5}
\usepackage{setspace}
\DeclareMathOperator{\vecc}{vec}
%\renewcommand{\vec}[1]{\ensuremath{\mathbf{#1}}}

\global\vbadness=1616

\begin{document}
\noindent
%%PART 1 of header
\begin{center}
	\vspace*{-3em}
	\def\arraystretch{2.0}
	\begin{tabular}{|p{0.7\linewidth}|p{0.2\linewidth}|}
		\hline
		\textbf{Avaliação de Matemática - Segundo Bimestre}                                                           & Pontos Obtidos $\downarrow$ \\
		\hline
		Data:\hspace{3cm}  Total de questões \textbf{\numquestions} \hspace{1cm} Total de pontos: \textbf{\numpoints} &                             \\
		\hline
		\multicolumn{2}{|l|}{Tuma: \hspace{0.3\linewidth} Nome: \hspace{0.3\linewidth} Duração: 1 hr}                                               \\
		\hline
	\end{tabular}
\end{center}
%%PART 2 of header, if you have too many questions, this may be a problem
%%if so, use \multirowgradetable{n}[questions], where n is the number of rows you want
%%or, switch to \gradetable[h][pages] instead,
\begin{center}
	\gradetable[h][questions]
\end{center}
%%PART 3 of header
\textbf{Instruções
	\begin{enumerate}
		\item Explique todas as questões claramente.
		\item Necessário todos os cálculos.
	\end{enumerate}
}
%%toggle comment on next line to show/hide the answers
% \printanswers
%%Now the actual paper!
\begin{questions}

	\question[1]

    Um construtor tem contratos para construir 3 estilos de casa: moderno, mediterrâneo e colonial. A
    quantidate de material empregada em cada tipo de casa é dada pela matriz:

	\begin{equation*}
		\begin{bmatrix}
            5 & 20 & 16 & 7 & 17 \\
            7 & 18 & 12 & 9 & 21 \\
            6 & 25 & 8 & 5 & 13
		\end{bmatrix}
	\end{equation*}

    Onde Linha 1 é estilo moderno, linha 2 estilo Mediterrâneo e linha 3 estilo Colonia.
    Coluna 1 é referente ao material de Ferro, coluna 2 é referente ao material de Madeira,
    coluna 3 é referente a Vidro, coluna 4 refere-se a Tinta e coluna 5 refere-se a Tijolo.

    Pergunta-se:

	\begin{tasks}
		\task Se ele vai construir 5, 7 e 12 casas dos tipos moderno, mediterrâneo
        e colonial, respectivamente, quantas unidades de cada material serão empregadas?
		\task Suponha agora que os preços por unidade de ferro, madeira, vidro, tinta e
        tijolo sejam, respectivamente, R\$ 15.00, R\$ 8.00, R\$ 5.00, R\$ 1.00, R\$ 10.00.
        Qual é o preço unitário de cada tipo de casa?
		\task Qual o custo total do material empregado?
	\end{tasks}

	\question[1]

	Determine a matriz X em cada uma das situações abaixo:

	\begin{tasks}(2)
		\task { $X - \begin{bmatrix}
					-4 & 5 \\
					-1 & 1 \\
					2 & -6
				\end{bmatrix} = \begin{bmatrix}
					5 & 0 \\
					2 & 3 \\
					7 & 8
				\end{bmatrix}$}
		\task { $4 \cdot \begin{bmatrix}
					-4 & 1 & -3  \\
					3 & -7 & 1 \\
				\end{bmatrix} + X = 5 \cdot \begin{bmatrix}
					3 & -2 & -1 \\
					3 & 2 & -1
				\end{bmatrix}$}
	\end{tasks}

	\question[1]

	Sendo as matrizes $A = \begin{bmatrix}
			1 & -2 & 2 \\
			-3 & 6 & 1
		\end{bmatrix}$, $B = \begin{bmatrix}
			2 & -1 \\
			5 & -4 \\
			3 & -2
		\end{bmatrix}$ e $C = \begin{bmatrix}
			3  & -2 & 5  \\
			-1 & 1  & -2
		\end{bmatrix}$, determine:

	\begin{tasks}
		\task $A \cdot B$
		\task $B \cdot C$
	\end{tasks}

	\question[1]

	A partir da matriz $A = (a_{ij})_{2 \times 2}$ tal que $a_{ij} = -3i + 4j$
	e $B = (b_{ij})_{2 \times 2}$ tal que $b_{ij} = 2i + 3j$,
	determine a matriz $A + B$.

	\question[1]

    Um professor aplica, durante os cinco dias úteis de uma semana, testes com quatro
    questões de múltipla escolha a cinco alunos. Os resultados foram representados
    na matriz:


	\begin{equation*}
		\begin{bmatrix}
            3 & 2 & 0 & 1 & 2 \\
            3 & 2 & 4 & 1 & 2 \\
            2 & 2 & 2 & 3 & 2 \\
            3 & 2 & 4 & 1 & 0 \\
            0 & 2 & 0 & 4 & 4
		\end{bmatrix}
	\end{equation*}

    Nessa matriz os elementos das linhas de 1 a 5 representam as quantidadeas de questões
    acertadas pelos alunos Ana, Bruno, Carlos, Denis e Érica, respectivamente, enquanto
    que as colunas de 1 a 5 indicam os dias da semana, de segunda-feira a sexta-feira,
    respectivamente, em que os testes foram aplicados.

    O teste que apresentou maior quantidade de acertos foi aplicado na:

	\begin{tasks}(5)
		\task segunda-feira
		\task terça-feira
		\task quarta-feira
		\task quinta-feira
		\task sexta-feira
	\end{tasks}



\end{questions}
\end{document}
