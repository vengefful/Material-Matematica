%%Template made by Uday Khankhoje for examinations using the exam template
%%Refer to the documentation http://www-math.mit.edu/~psh/exam/examdoc.pdf
%%for lot more bells and whistles to the standard template shown below
\documentclass[a4paper,11pt,addpoints]{exam}
\usepackage[left=1.5cm,right=1.5cm,top=1.5cm,bottom=2cm]{geometry}
%\usepackage{mathrsfs}
\usepackage{graphicx,color}
\usepackage[x11names]{xcolor}
\usepackage{venndiagram}
\usepackage{epic,eepic}
%\usepackage{mathpazo}
\usepackage{url}
\usepackage{tasks} % cria lista curta
\usepackage{multicol}
\usepackage{amsmath, amsthm, amssymb}
\pointsinmargin
\boxedpoints
\renewcommand*\half{.5}
\usepackage{setspace}
\DeclareMathOperator{\vecc}{vec}
%\renewcommand{\vec}[1]{\ensuremath{\mathbf{#1}}}

\global\vbadness=1616

\begin{document}
\noindent
%%PART 1 of header
\begin{center}
	\vspace*{-3em}
	\def\arraystretch{2.0}
	\begin{tabular}{|p{0.7\linewidth}|p{0.2\linewidth}|}
		\hline
		\textbf{Avaliação de Matemática - Segundo Bimestre}                                                           & Pontos Obtidos $\downarrow$ \\
		\hline
		Data:\hspace{3cm}  Total de questões \textbf{\numquestions} \hspace{1cm} Total de pontos: \textbf{\numpoints} &                             \\
		\hline
		\multicolumn{2}{|l|}{Tuma: \hspace{0.3\linewidth} Nome: \hspace{0.3\linewidth} Duração: 1 hr}                                               \\
		\hline
	\end{tabular}
\end{center}
%%PART 2 of header, if you have too many questions, this may be a problem
%%if so, use \multirowgradetable{n}[questions], where n is the number of rows you want
%%or, switch to \gradetable[h][pages] instead,
\begin{center}
	\gradetable[h][questions]
\end{center}
%%PART 3 of header
\textbf{Instruções
	\begin{enumerate}
		\item Explique todas as questões claramente.
		\item Necessário todos os cálculos.
	\end{enumerate}
}
%%toggle comment on next line to show/hide the answers
% \printanswers
%%Now the actual paper!
\begin{questions}

	\question[1]

	Quatro seleções (Rússia, Itália, Brasil e Estados Unidos) disputaram a
	etapa final de um torneio internacional de vôlei no sistema ``todos jogam
	contra todos'' uma única vez. O campeão do torneio será a equipe que obtiver
	mais vitórias; em caso de empate no número de vitórias, o campeão é
	decidido pelo resultado obtido no confronto direto entre as equipes
	empatadas. Na matriz seguinte, o elemento $a_{ij}$ indica o número de
	\textit{sets} que a seleção \textit{i} venceu no jogo contra a seleção
	\textit{j}.

	\begin{equation*}
		\begin{bmatrix}
			0 & 2 & 3 & 1 \\
			3 & 0 & 1 & 3 \\
			2 & 3 & 0 & 3 \\
			3 & 2 & 0 & 0
		\end{bmatrix}
	\end{equation*}

	Lembre-se que o jogo de vôlei termina quando uma equipe complete três sets.

	Representando Rússia pela Linha 1, Itália pela linha 2, Brasil pela linha 3
	e EUA pela linha 4, determine:

	\begin{tasks}
		\task O número de vitórias da equipe norte-americana;
		\task O placar do jogo Brasil x Itália;
		\task O número de \textit{sets} marcados contra a Rússia;
		\task O campeão do torneio.
	\end{tasks}

	\question[1]

	Determine a matriz X em cada uma das situações abaixo:

	\begin{tasks}(2)
		\task { $X + \begin{bmatrix}
					4 & 3 \\
					1 & 1 \\
					2 & 0
				\end{bmatrix} = \begin{bmatrix}
					5 & 0 \\
					2 & 3 \\
					7 & 8
				\end{bmatrix}$}
		\task { $2 \cdot \begin{bmatrix}
					4 & 1 & 3  \\
					6 & 2 & -1 \\
				\end{bmatrix} + X = 3 \cdot \begin{bmatrix}
					1 & 2 & -1 \\
					0 & 2 & 1
				\end{bmatrix}$}
	\end{tasks}

	\question[1]

	Sendo as matrizes $A = \begin{bmatrix}
			1 & 2 & 3 \\
			0 & 6 & 1
		\end{bmatrix}$, $B = \begin{bmatrix}
			1 & 1 \\
			4 & 4 \\
			2 & 2
		\end{bmatrix}$ e $C = \begin{bmatrix}
			1  & -1 & 1  \\
			-1 & 1  & -1
		\end{bmatrix}$, determine:

	\begin{tasks}
		\task $A \cdot B$
		\task $B \cdot C$
	\end{tasks}

	\question[1]

	A partir da matriz $A = (a_{ij})_{2 \times 2}$ tal que $a_{ij} = 3i + 2j$
	e $B = (b_{ij})_{2 \times 2}$ tal que $b_{ij} = i + j$,
	determine a matriz $A + B$.

	\question[1]

	Uma construtora, pretendendo investir na construção de imóveis em uma
	metrópole com cinco grandes regiões, fez uma pesquisa sobre a quantidade
	de famílias que mudaram de uma região para outra, de modo a determinar
	qual região foi o destino do maior fluxo de famílias, sem levar em
	consideração o número de famílias que deixaram a região. Os valores da
	pesquisa estão dispostos em uma matriz $A = (a_{ij})$, $i,j \in \{1, 2, 3, 4, 5\}$,
	em que o elemento $a_{ij}$ corresponde ao total de famílias (em dezena)
	que se mudaram da região \textit{i} para a região \textit{j} durante
	um certo período, e o elemento a é considerado nulo, uma vez que somente
	são consideradas mudanças entre regiões distintas. A seguir, está
	apresentada a matriz com os dados da pesquisa.

	\begin{equation*}
		\begin{bmatrix}
			0 & 4 & 2 & 2 & 5 \\
			0 & 0 & 6 & 2 & 3 \\
			2 & 2 & 0 & 3 & 0 \\
			1 & 0 & 2 & 0 & 4 \\
			1 & 2 & 0 & 4 & 0
		\end{bmatrix}
	\end{equation*}

	Qual região foi selecionada para o investimento da construtora?

	\begin{tasks}(5)
		\task 1
		\task 2
		\task 3
		\task 4
		\task 5
	\end{tasks}



\end{questions}
\end{document}
