% TeX root = main.tex
\section{Introdução}

\begin{figure}[htb!]
	\centering
	\includegraphics[width=.6\linewidth]{images/quadro.png}
	\caption{expectativa de vida brasileira em 2008}
	\label{fig:}
\end{figure}

Note que podemos encontrar a expectativa de vida de uma mulher residente na região Sul bastando olhar o cruzamento da linha 2 com a coluna 4, onde encontramos o valor de 78,5 anos.

Em matemática, as tabelas como essa são chamadas de \textbf{matrizes}, sobre as quais definiremos a relação de igualdade e algumas operações.

\section{Definição}
\dfn{Matriz}{Chama-se \textbf{matriz do tipo $m \times n$} (lemos ``m por n'') toda tabela de números dispostos em \textit{m} linhas e \textit{n} colunas.
}

Essa tabela deve ser representada entre parênteses () ou entre colchetes [].

\begin{examples}\leavevmode

	\begin{tasks}
		\task{
			$\begin{bmatrix}
					-6 & 7  \\
					-4 & 0  \\
					2  & -1 \\
				\end{bmatrix}$
			é uma matriz do tipo $3 \times 2$, pois tem 3 linhas e 2 colunas.
		}
		\task {
			$\begin{bmatrix}
					3 & \sqrt{2} & -5
				\end{bmatrix}$
			é uma matriz do tipo $1 \times 3$, pois tem 1 linha e 3 colunas.
		}
	\end{tasks}

\end{examples}

\section{Representação genérica}

Indicamos por \textbf{$a_{ij}$} o elemento posicionado na linha \textit{i} e na coluna \textit{j} de uma matriz \textit{A}.

Na matriz:

\begin{equation*}
	A_{3 \times 2} = \begin{bmatrix}
		6  & 7  \\
		-4 & 0  \\
		2  & -1
	\end{bmatrix}
\end{equation*}

\begin{itemize}
	\item o elemento 6 está na linha 1 e na coluna 1; por isso, ele é indicado por $a_{11}$, ou seja, $a_{11} = 6$;
	\item o elemento 7 está na linha 1 e na coluna 2; por isso, ele é indicado por $a_{12}$, ou seja, $a_{12} = 7$;
	\item analogamente, temos $a_{21} = -4$, $a_{22} = 0$, $a_{31} = 2$, $a_{32} = -1$.
\end{itemize}

\dfn{Matriz Genérica}{
	Representamos genericamente uma matriz \textit{A} do tipo $m \times n$ da seguinte maneira:

	\begin{equation*}
		A_{m \times n} = \begin{bmatrix}
			a_{11} & a_{12} & a_{13} & \dots  & a_{1n} \\
			a_{21} & a_{22} & a_{23} & \dots  & a_{2n} \\
			\vdots & \vdots & \vdots & \vdots & \vdots \\
			a_{m1} & a_{m2} & a_{m3} & \dots  & a_{mn} \\
		\end{bmatrix}
	\end{equation*}
}
Como essa representação é muita extensa, vamos convencionar uma forma abreviada. Essa matriz pode ser representada simplesmente por
$A = (a_{ij})_{m \times n}$ ou, quando não houver possibilidade de confusão quanto ao tipo de matriz, por $A = (a_{ij})$.

\begin{exercise}
	Representar explicitamente a matriz $A = (a_{ij})_{2 \times 4}$ tal que $a_{ij} = 2i + j$.

	Primeiro, representamos genericamente a matriz \textit{A}, do tipo $2 \times 4$:
	\begin{equation*}
		A = \begin{bmatrix}
			a_{11} & a_{12} & a_{13} & a_{14} \\
			a_{21} & a_{22} & a_{23} & a_{24} \\
		\end{bmatrix}
	\end{equation*}

	A seguir, calculamos o valor de cada elemento $a_{ij}$, pela lei $a_{ij} = 2i + j$:

	\begin{tasks}(2)
		\task[] $a_{11} = 2 \cdot 1 + 1 = 3$
		\task[] $a_{12} = 2 \cdot 1 + 2 = 4$
		\task[] $a_{13} = 2 \cdot 1 + 3 = 5$
		\task[] $a_{14} = 2 \cdot 1 + 4 = 6$
		\task[] $a_{21} = 2 \cdot 2 + 1 = 5$
		\task[] $a_{22} = 2 \cdot 2 + 2 = 6$
		\task[] $a_{23} = 2 \cdot 2 + 3 = 7$
		\task[] $a_{24} = 2 \cdot 2 + 4 = 8$
	\end{tasks}

	Concluindo, temos a matriz: $A = \begin{bmatrix}
			3 & 4 & 5 & 6 \\
			5 & 6 & 7 & 8
		\end{bmatrix}$
\end{exercise}

\section{Matrizes Especiais}

\subsection{Matriz Quadrada}

\dfn{Matriz Quadrada}{É toda matriz cujo número de linhas é igual ao número de colunas.}

O número de linhas ou de colunas de uma matriz quadrada é chamado de \textbf{ordem} da matriz.

\begin{examples}\leavevmode
	\begin{tasks}(2)
		\task $\begin{bmatrix}
				4  & 9 & 0  \\
				-6 & 2 & 4  \\
				3  & 5 & -2
			\end{bmatrix}$
		é uma matriz quadrada de ordem 3.

		\task $\begin{bmatrix}
				3 & -9 \\
				0 & 1
			\end{bmatrix}$
		é uma matriz quadrada de ordem 2.
	\end{tasks}
\end{examples}

Numa matriz A de ordem \textit{n}, os elementos $a_{ij}$, tais que $i = j$ formam a \textbf{diagonal principal} da matriz, e os elementos
$a_{ij}$, tais que $i + j = n + 1$ formam a \textbf{diagonal secundária}. Por exemplo:

\begin{figure}[htb!]
	\centering
	\includegraphics[width=.3\linewidth]{images/quadrada.png}
\end{figure}

\subsection{Matriz Identidade}

\dfn{Matriz Identidade}{É a matriz quadrada cujos elementos da \textbf{diagonal principal} são iguais a 1 e os demais iguais a 0.}

Indicamos por $I_n$ a matriz identidade de ordem \textit{n}.

\begin{examples}\leavevmode
	\begin{tasks}(2)
		\task $I_3 = \begin{bmatrix}
				1 & 0 & 0 \\
				0 & 1 & 0 \\
				0 & 0 & 1
			\end{bmatrix}$

		\task $I_2 = \begin{bmatrix}
				1 & 0 \\
				0 & 1
			\end{bmatrix}$
	\end{tasks}
\end{examples}

\subsection{Matriz Nula}

\dfn{Matriz Nula}{É a matriz que possui todos os elementos iguais a zero.}

\begin{examples}\leavevmode
	\begin{tasks}(2)
		\task
		$\begin{bmatrix}
				0 & 0 & 0 \\
				0 & 0 & 0 \\
				0 & 0 & 0 \\
			\end{bmatrix}$
		\task $\begin{bmatrix}
				0 & 0 \\
				0 & 0 \\
			\end{bmatrix}$
	\end{tasks}

\end{examples}

\subsection{Transposta de uma Matriz}

\dfn{Transposta de uma Matriz}{Transposta de uma matriz \textit{A} é a matriz $A^t$ tal que os números que a formam são obtidos através da troca de posição entre linhas e colunas da matriz \textit{A}. }

\begin{examples}\leavevmode
	\begin{tasks}
		\task {A transposta de $A_{3 \times 2} = \begin{bmatrix}
				5 & -4 \\
				6 & 2  \\
				0 & 7
			\end{bmatrix}$
		é a matriz $A^t_{2 \times 3} = \begin{bmatrix}
				5  & 6 & 0 \\
				-4 & 2 & 7 \\
			\end{bmatrix}$
		}

		\task {A transposta de $B_{1 \times 4} = \begin{bmatrix}
				2 & 0 & -5 & 8 \\
			\end{bmatrix}$
		é a matriz $B^t_{4 \times 1} = \begin{bmatrix}
				2  \\
				0  \\
				-5 \\
				8
			\end{bmatrix}$
		}
	\end{tasks}
\end{examples}

Nome que a transposta de uma matriz $m \times n$ é uma matriz do tipo $n \times m$.

\subsection{Igualdade de Matrizes}
\dfn{Igualdade de Matrizes}{
	Duas Matrizes do mesmo tipo são iguais quando todos os elementos correspondentes são iguais.
}

\begin{exercise}
	Determinar o número real \textit{x} tal que: $\begin{bmatrix}
			6 & x^2-5 \\
			0 & x
		\end{bmatrix} = \begin{bmatrix}
			6 & 11 \\
			0 & 4
		\end{bmatrix}$

	\vspace{.3cm}
	\textbf{Resolução} \vspace{.3cm}

	As matrizes são do mesmo tipo $(2 \times 2)$. Logo, elas serão iguais se, e somente se, os elementos
	correspondentes forem iguais, isto é:

	\begin{equation*}
		\begin{cases}
			6 = 6        \\
			x^2 - 5 = 11 \\
			0 = 0        \\
			x = 4
		\end{cases}
		\implies
		\begin{cases}
			x^2 = 16 \\
			x = 4
		\end{cases}
		\therefore
		\begin{cases}
			x = \pm 4 \\
			x = 4
		\end{cases}
	\end{equation*}

	Como o número $4$ é a única solução comum às duas equações do sistema, concluímos que as matrizes são iguais se, e
	somente se, $x = 4$.

\end{exercise}

\subsection{Exercícios Propostos}

\begin{enumerate}[label*=\protect\fbox{\arabic{enumi}}]
	\item {
	      Uma rede comercial é formada por cinco lojas, numeradas de 1 a 5. A tabela abaixo mostra o faturamento, em real, de
	      cada loja nos quatro primeiros dias de janeiro:

	      \begin{equation*}
		      \begin{split}
			      \begin{bmatrix}
				      1950 & 2030 & 1800 & 1950 \\
				      1500 & 1820 & 1740 & 1680 \\
				      3010 & 2800 & 2700 & 3050 \\
				      2500 & 2420 & 2300 & 2680 \\
				      1800 & 2020 & 2040 & 1950
			      \end{bmatrix}
		      \end{split}
	      \end{equation*}

	      Cada elemento $a_{ij}$ dessa matriz é o faturamento da loja \textit{i} no dia \textit{j}.

	      \begin{tasks}
		      \task Qual foi o faturamento da loja 3 no dia 2?
		      \task Qual foi o faturamento dessa rede de lojas no dia 3?
		      \task Qual foi o faturamento da loja 1 nos quatro dias?
	      \end{tasks}
	      }
	\item {
	      Represente explicitamente cada uma das matrizes:
	      \begin{tasks}
		      \task $A = (a_{ij})_{3 \times 2}$ tal que $a_{ij} = i + 2j$
		      \task $B = (b_{ij})_{2 \times 3}$ tal que $b_{ij} = i^2 + 3j$
		      \task $C = (c_{ij})_{2 \times 2}$ tal que $c_{ij} = 2i$
		      \task $D = (a_{ij})_{2 \times 3}$ tal que $\begin{cases} 1, \text{ se } i = j \\ i + j, \text{ se } i \neq j \end{cases}$
	      \end{tasks}
	      }

	\item {
	      Sendo $I_2$ a matriz identidade de ordem 2, determine o número real \textit{x} tal que:
	      \begin{equation*}
		      \begin{split}
			      \begin{bmatrix}
				      x^2 - 15 & 0     \\
				      0        & x - 3
			      \end{bmatrix} = I_2
		      \end{split}
	      \end{equation*}
	      }

	\item {
	      Dada a matriz $A = \begin{bmatrix}
			      5  & 4 & -2 \\
			      -6 & 0 & 3
		      \end{bmatrix}$,
	      determine as matrizes:
	      \begin{tasks}
		      \task $A^t$
		      \task $(A^t)^t$
	      \end{tasks}
	      }

	\item {
	      Obtenha os valores reais de \textit{x} e \textit{y} de modo que a matriz abaixo seja nula.
	      \begin{equation*}
		      \begin{split}
			      \begin{bmatrix}
				      3x + y - 7 & 0          & 0 \\
				      0          & 5x - y - 1 & 0
			      \end{bmatrix}
		      \end{split}
	      \end{equation*}
	      }
\end{enumerate}


\section{Operações entre Matrizes}

\subsection{Adição de matrizes}

\dfn{Adição de Matrizes}{
	A \textbf{soma} de duas matrizes do mesmo tipo, \textit{A} e \textit{B}, é a matriz em que cada elemento é a soma de seus
	correspondentes em \textit{A} e \textit{B}.

}

Indicamos essa soma por: $A + B$.
\begin{example}
	\begin{equation*}
		\begin{bmatrix}
			4  & 7 \\
			-5 & 3
		\end{bmatrix}
		+ \begin{bmatrix}
			2  & 1  \\
			-3 & -6
		\end{bmatrix}
		= \begin{bmatrix}
			6  & 8  \\
			-8 & -3
		\end{bmatrix}
	\end{equation*}
\end{example}

\begin{proposition}{Adição de Matrizes}{trigsquare}
	\begin{enumerate}
		\item Associativa: $(A + B) + C = A + (B + C) = A + B + C$
		\item Comutativa: $A + B = B + A$
		\item Elemento Neutro: $A = 0 = 0 + A = A$. Onde $0$ é a \textbf{matriz nula}.
		\item Elemento oposto: $A + (-A) = (-A) + A = 0$. Onde $-A$ é a \textbf{matriz oposta} de \textit{A}.
	\end{enumerate}
\end{proposition}

\begin{example}
	Dado $A = \begin{bmatrix}
			2 & 0  \\
			7 & -8
		\end{bmatrix}$,
	sua \textbf{matriz oposta} é $-A = \begin{bmatrix}
			-2 & 0 \\
			-7 & 8
		\end{bmatrix}$, pois:
	\begin{equation*}
		A + (-A) = \begin{bmatrix}
			2 & 0  \\
			7 & -8
		\end{bmatrix} + \begin{bmatrix}
			-2 & 0 \\
			-7 & 8
		\end{bmatrix} = \begin{bmatrix}
			0 & 0 \\
			0 & 0
		\end{bmatrix}
	\end{equation*}
\end{example}

\subsection{Subtração de Matrizes}
\dfn{Subtração de Matrizes}{
	A \textbf{diferença} de duas matrizes do mesmo tipo, \textit{A} e \textit{B}, nessa ordem, é a soma de \textit{A} com
	a oposta de \textit{B}.

}

Indicamos essa diferença por $A - B$.
\begin{example}
	Sendo $A = \begin{bmatrix}
			9  & 6  \\
			4  & 0  \\
			-4 & -1
		\end{bmatrix}$ e $B = \begin{bmatrix}
			2  & 4  \\
			-3 & 5  \\
			1  & -1 \\
		\end{bmatrix}$, temos:

	\begin{equation*}
		A - B = A + (-B) = \begin{bmatrix}
			9  & 6  \\
			4  & 0  \\
			-4 & -1
		\end{bmatrix} + \begin{bmatrix}
			-2 & -4 \\
			3  & -5 \\
			-1 & 1
		\end{bmatrix} = \begin{bmatrix}
			7  & 2  \\
			7  & -5 \\
			-5 & 0
		\end{bmatrix}
	\end{equation*}

	Para simplificar esse procedimento, podemos subtrair os elementos correspondentes em \textit{A} e \textit{B}:
	\begin{equation*}
		A - B = \begin{bmatrix}
			9  & 6  \\
			4  & 0  \\
			-4 & -1
		\end{bmatrix} - \begin{bmatrix}
			2  & 4  \\
			-3 & 5  \\
			1  & -1
		\end{bmatrix} = \begin{bmatrix}
			9 -2     & 6 - 4     \\
			4 - (-3) & 0 - 5     \\
			-4 -1    & -1 - (-1)
		\end{bmatrix} = \begin{bmatrix}
			7  & 2  \\
			7  & -5 \\
			-5 & 0
		\end{bmatrix}
	\end{equation*}
\end{example}

\subsection{Multiplicação de um número real por uma Matriz}

\dfn{Multiplicação de um número real por uma Matriz}{
	O \textbf{produto} de um número real \textit{k} por uma matriz \textit{A} é a matriz em que cada elemento é o
	produto de seu correspondente em \textit{A} pelo número \textit{k}.

}

Indicamos esse produto por $k \cdot A \text{ ou } kA $.
\begin{example}
	\[6 \cdot \begin{bmatrix}
			2 & 5        & 1  \\
			0 & \sqrt{2} & -2
		\end{bmatrix} = \begin{bmatrix}
			12 & 30        & 6   \\
			0  & 6\sqrt{2} & -12
		\end{bmatrix}\]
\end{example}

\subsection{Exercícios Propostos}

\begin{multicols}{2}
	\begin{enumerate}[label*=\protect\fbox{\arabic{enumi}}]
		\item {
		      Dadas as matrizes $A = \begin{bmatrix}
				      2 & 3  & 8 \\
				      1 & -4 & 0
			      \end{bmatrix}$, $B = \begin{bmatrix}
				      4 & 5 & -9 \\
				      6 & 2 & 7
			      \end{bmatrix}$ e $C = \begin{bmatrix}
				      2  & 0  \\
				      8  & 6  \\
				      -4 & 10
			      \end{bmatrix}$ determine:
		      \begin{tasks}
			      \task $A + B$
			      \task $2A - B$
			      \task $3A - \frac{1}{2}\cdot C^t$
		      \end{tasks}
		      }

		\item {
		      Determine a matriz \textit{X} tal que:
		      \begin{equation*}
			      2 \cdot \begin{bmatrix}
				      4 & 1 & 3  \\
				      6 & 2 & -1
			      \end{bmatrix} + X = 3 \cdot \begin{bmatrix}
				      1 & 2 & -1 \\
				      0 & 2 & 1
			      \end{bmatrix}
		      \end{equation*}
		      }

		\item {
		      Determine as matrizes \textit{X} e \textit{Y} tais que:
		      \begin{equation*}
			      X + Y = \begin{bmatrix}
				      0  & 1 \\
				      -6 & 4
			      \end{bmatrix} \text{ e } X - Y = \begin{bmatrix}
				      2 & -7 \\
				      4 & 6
			      \end{bmatrix}
		      \end{equation*}
		      }
	\end{enumerate}
\end{multicols}

\subsection{Multiplicação de Matrizes}

\dfn{Multiplicação de Matrizes}{
	O \textbf{produto} da matriz $A = (a_{ij})_{m \times n}$ pela matriz $B = (b_{ij})_{n \times p}$ é a matriz
	$C = (c_{ij})_{m \times p}$ tal que cada elemento $c_{ij}$ é o produto da linha \textit{i} de \textit{A} pela coluna
	\textit{j} de \textit{B}.
}

Esse produto é indicado por $A \cdot B$ ou $AB$.
O esquema a seguir ajuda a visualizar essa definição:

\vspace{.5cm}

% l' unite
\newcommand{\myunit}{1 cm}
\tikzset{
	node style sp/.style={draw,circle,minimum size=\myunit},
	node style ge/.style={circle,minimum size=\myunit},
	arrow style mul/.style={draw,sloped,midway,fill=white},
	arrow style plus/.style={midway,sloped,fill=white},
}

\begin{center}
	\begin{tikzpicture}[>=latex]
		% les matrices
		\matrix (A) [matrix of math nodes,
				nodes = {node style ge},
				left delimiter  = (,
				right delimiter = )] at (0,0)
		{
		a_{11} & a_{12} & \ldots & a_{1p}  \\
		|[node style sp]| a_{21}
		& |[node style sp]| a_{22}
		& \ldots
		& |[node style sp]| a_{2p} \\
		\vdots & \vdots & \ddots & \vdots  \\
		a_{n1} & a_{n2} & \ldots & a_{np}  \\
		};
		\node [draw,below=10pt] at (A.south)
		{ $A$ : \textcolor{red}{$n$ linhas} $p$ colunas};

		\matrix (B) [matrix of math nodes,
				nodes = {node style ge},
				left delimiter  = (,
				right delimiter = )] at (6*\myunit,6*\myunit)
		{
		b_{11} & |[node style sp]| b_{12}
		& \ldots & b_{1q}  \\
		b_{21} & |[node style sp]| b_{22}
		& \ldots & b_{2q}  \\
		\vdots & \vdots & \ddots & \vdots  \\
		b_{p1} & |[node style sp]| b_{p2}
		& \ldots & b_{pq}  \\
		};
		\node [draw,above=10pt] at (B.north)
		{ $B$ : $p$ linhas \textcolor{red}{$q$ colunas}};
		% matrice résultat
		\matrix (C) [matrix of math nodes,
				nodes = {node style ge},
				left delimiter  = (,
				right delimiter = )] at (6*\myunit,0)
		{
		c_{11} & c_{12} & \ldots & c_{1q} \\
		c_{21} & |[node style sp,red]| c_{22}
		& \ldots & c_{2q} \\
		\vdots & \vdots & \ddots & \vdots \\
		c_{n1} & c_{n2} & \ldots & c_{nq} \\
		};
		% les fleches
		\draw[blue] (A-2-1.north) -- (C-2-2.north);
		\draw[blue] (A-2-1.south) -- (C-2-2.south);
		\draw[blue] (B-1-2.west)  -- (C-2-2.west);
		\draw[blue] (B-1-2.east)  -- (C-2-2.east);
		\draw[<->,red](A-2-1) to[in=180,out=90]
		node[arrow style mul] (x) {$a_{21}\times b_{12}$} (B-1-2);
		\draw[<->,red](A-2-2) to[in=180,out=90]
		node[arrow style mul] (y) {$a_{22}\times b_{22}$} (B-2-2);
		\draw[<->,red](A-2-4) to[in=180,out=90]
		node[arrow style mul] (z) {$a_{2p}\times b_{p2}$} (B-4-2);
		\draw[red,->] (x) to node[arrow style plus] {$+$} (y)%
		to node[arrow style plus] {$+\raisebox{.5ex}{\ldots}+$} (z)
		to (C-2-2.north west);


		\node [draw,below=10pt] at (C.south)
		{$ C=A\times B$ : \textcolor{red}{$n$ linhas}
			\textcolor{red}{$q$ colunas}};

	\end{tikzpicture}
\end{center}

\begin{examples}\leavevmode
	\begin{enumerate}
		\item {
		      $\begin{bmatrix}
				      2 & 1 & 3
			      \end{bmatrix} \cdot \begin{bmatrix}
				      5 & 3  \\
				      2 & 0  \\
				      1 & -2
			      \end{bmatrix} = \begin{bmatrix}
				      2 \cdot 5 + 1 \cdot 2 + 3 \cdot 1 & 2 \cdot 3 + 1 \cdot 0 + 3 \cdot (-2)
			      \end{bmatrix} = \begin{bmatrix}
				      15 & 0
			      \end{bmatrix}$
		      }

		\item {
		      $\begin{bmatrix}
				      3 & 5 \\
				      2 & 0
			      \end{bmatrix} \cdot  \begin{bmatrix}
				      4 & 5 & -2 \\
				      2 & 1 & 3
			      \end{bmatrix} = \begin{bmatrix}
				      3 \cdot 4 + 5 \cdot 2 & 3 \cdot 5 + 5 \cdot 1 & 3 \cdot (-2) + 5 \cdot 3 \\
				      2 \cdot 4 + 0 \cdot 2 & 2 \cdot 5 + 0 \cdot 1 & 2 \cdot (-2) + 0 \cdot 3
			      \end{bmatrix} = \begin{bmatrix}
				      22 & 20 & 9  \\
				      8  & 10 & -4
			      \end{bmatrix}$
		      }
	\end{enumerate}
\end{examples}

\begin{note}
	\begin{enumerate}
		\item Se \textit{A} e \textit{B} são matrizes, existe o produto $AB$ se, e somente se, o número de colunas
		      de \textit{A} é igual ao número de linhas de \textit{B}. Veja abaixo:
		      \begin{tasks}(2)
			      \task {
				      Existe o produto $A_{3 \times 4} \cdot B_{4 \times 5}$
			      }
			      \task Não existe o produto $A_{2 \times 3} \cdot B_{4 \times 2}$
		      \end{tasks}
		\item {
		      A matriz \textit{C}, tal que $C = AB$, possui o mesmo número de linhas de \textit{A} e o mesmo número de colunas
		      de \textit{B}, isto é:
		      \begin{equation*}
			      A_{m \times k} \cdot B_{k \times n} = C_{m \times n}
		      \end{equation*}

		      Por exemplo:

		      \begin{tasks}(2)
			      \task $A_{3 \times 5} \cdot B_{5 \times 8} = C_{3 \times 8}$
			      \task $A_{1 \times 4} \cdot B_{4 \times 1} = C_{1 \times 1}$
		      \end{tasks}
		      }
	\end{enumerate}
\end{note}

\begin{proposition}{Propriedades da multiplicação de matrizes}{}
	\begin{enumerate}
		\item Associativa: $(A \cdot B) \cdot C = A \cdot (B \cdot C) = A \cdot B \cdot C$, em que $A_{m \times n}$, $B_{n \times k}$ e $C_{k \times p}$.
		\item Distributiva: $(A + B) \cdot C = A \cdot C + B \cdot C$, em que $A_{m \times n}$, $B_{m \times n}$ e $C_{k \times m}$.
		\item Elemento neutro: $A \cdot I_n = A$ e $I_m \cdot A = A$
		\item Transposta do produto: $(A \cdot B)^t = B^t \cdot A^t$, em que $A_{m \times n}$ e $B_{n \times k}$.
	\end{enumerate}
\end{proposition}

\begin{exercise}\leavevmode
	\begin{enumerate}
		\item {
		      Determinar a matriz \textit{X} tal que: $\begin{bmatrix}
				      2 & 3  \\
				      1 & -4
			      \end{bmatrix} \cdot X = \begin{bmatrix}
				      4 \\
				      -9
			      \end{bmatrix}$

		      \vspace{.3cm}
		      \textbf{Resolução}
		      \vspace{.3cm}

		      Primeiro, vamos determinar o tipo da matriz \textit{X}:

		      \begin{equation*}
			      \begin{bmatrix}
				      2 & 3  \\
				      1 & -4
			      \end{bmatrix}_{2 \times 2} \cdot X_{m \times n} = \begin{bmatrix}
				      4 & -9
			      \end{bmatrix}_{2 \times 1}
		      \end{equation*}

		      Para que seja possível multiplicar as matrizes, o número de colunas da primeira
		      matriz deve ser igual ao número de linhas de \textit{X}; portanto, $m = 2$. O número
		      de colunas da matriz \textit{X} deve ser igual ao número de colunas da matriz produto;
		      portanto, $n = 1$.

		      Assim, a matriz \textit{X} é to tipo $2 \times 1$.

		      Sendo $X = \begin{bmatrix}
				      a \\ b
			      \end{bmatrix}$, temos:

		      \begin{equation*}
			      \begin{bmatrix}
				      2 & 3  \\
				      1 & -4
			      \end{bmatrix} \cdot \begin{bmatrix}
				      a \\
				      b
			      \end{bmatrix} = \begin{bmatrix}
				      4 \\
				      -9
			      \end{bmatrix} \implies \begin{bmatrix}
				      2a + 3b \\
				      a - 4b
			      \end{bmatrix} = \begin{bmatrix}
				      4 \\
				      -9
			      \end{bmatrix}
		      \end{equation*}

		      Portanto: \[
			      \systeme{
				      2a + 3b = 4,
				      a - 4b = -9
			      } \implies
			      \systeme{
				      2a + 3b = 4,
				      -2a + 8b = 18
			      } \implies
			      0a + 11b = 22 \implies b = 2 \implies a = -1
		      \]

		      Assim, concluímos: $X = \begin{bmatrix}
				      -1 \\
				      2
			      \end{bmatrix}$
		      }
	\end{enumerate}
\end{exercise}

\subsection{Exercícios Propostos}

\begin{enumerate}[label*=\protect\fbox{\arabic{enumi}}]
	\item{ Dadas as matrizes $A = \begin{bmatrix}
			            2  & 6 \\
			            -1 & 0
		            \end{bmatrix}$, $B = \begin{bmatrix}
			            4 \\
			            3
		            \end{bmatrix}$ e $C = \begin{bmatrix}
			            1 & -2
		            \end{bmatrix}$, determine, se possível:

	            \begin{tasks}(3)
		            \task $A \cdot B$
		            \task $A \cdot C$
		            \task $B \cdot C$
		            \task $A^2$
		            \task $B^2$
	            \end{tasks}
	      }
	\item {
	      Sendo as matrizes $A = \begin{bmatrix}
			      1 & 2 & 3 \\
			      0 & 6 & 1
		      \end{bmatrix}$, $B = \begin{bmatrix}
			      1 & 1 \\
			      4 & 4 \\
			      2 & 2
		      \end{bmatrix}$ e $C = \begin{bmatrix}
			      1  & -1 & 1  \\
			      -1 & 1  & -1
		      \end{bmatrix}$, determine:

	      \begin{tasks}(3)
		      \task $A \cdot B$
		      \task $B \cdot A$
		      \task $A \cdot I_3$
		      \task $I_2 \cdot A$
		      \task $B \cdot C$
	      \end{tasks}
	      }

	      \item{
	                  O valor de \textit{a} para que a setença $\begin{bmatrix}
			                  2 & 1 \\
			                  1 & 1
		                  \end{bmatrix} \cdot  \begin{bmatrix}
			                  1  & -1 \\
			                  -1 & a
		                  \end{bmatrix} = \begin{bmatrix}
			                  1 & 0 \\
			                  0 & 1
		                  \end{bmatrix}$ seja verdadeira é:

	                  \begin{tasks}(3)
		                  \task 1
		                  \task 2
		                  \task 0
		                  \task -2
		                  \task -1
	                  \end{tasks}
	            }

	\item {
	      Dadas as matrizes $A = (a_{ij})_{9 \times 8}$, com $a_{ij} = 2j$; $B = (b_{ij})_{8 \times 6}$, com $b_{ij} = i$; e $C \cdot A$,
	      determine o elemento $C_{45}$ da matriz \textit{C}.
	      }

	\item {
	      Dadas as matrizes $A = \begin{bmatrix}
			      1 & 2  \\
			      0 & -3
		      \end{bmatrix}$, $B = \begin{bmatrix}
			      6 \\
			      -15
		      \end{bmatrix}$, obtenha a matriz \textit{X} tal que $A \cdot X = B$.
	      }
\end{enumerate}

\subsection{Matrizes Inversas}

\dfn{Matrizes Inversas}{
	Uma matriz \textit{A} de ordem \textit{n} é \textbf{invertível} se, e somente se,
	existe uma matriz \textit{B} tal que:
	\begin{equation*}
		AB = BA = I_n
	\end{equation*}
	em que $I_n$ é a matriz identidade de ordem \textit{n}.
}

\begin{example}
	As matrizes $A = \begin{bmatrix}
			1 & 1 \\
			3 & 4
		\end{bmatrix}$ e $B = \begin{bmatrix}
			4  & -1 \\
			-3 & 1
		\end{bmatrix}$ são inversas entre si, pois:
	\begin{equation*}
		AB = \begin{bmatrix}
			1 & 1 \\
			3 & 4
		\end{bmatrix} \cdot \begin{bmatrix}
			4  & -1 \\
			-3 & 1
		\end{bmatrix} = \begin{bmatrix}
			1 & 0 \\
			0 & 1 \\
		\end{bmatrix} = I_2 \text{  e  } BA = \begin{bmatrix}
			4  & -1 \\
			-3 & 1
		\end{bmatrix} \cdot \begin{bmatrix}
			1 & 1 \\
			3 & 4
		\end{bmatrix} = \begin{bmatrix}
			1 & 0 \\
			0 & 1
		\end{bmatrix} = I_2
	\end{equation*}
	Assim, indicamos $B = A^{-1}$ ou, de maneira equivalente, $A = B^{-1}$.
\end{example}

\begin{exercise}\leavevmode
	Determinar, se existir, a inversa de cada uma das matrizes.

	\begin{tasks}(2)
		\task $A = \begin{bmatrix}
				1 & 3 \\
				0 & 2
			\end{bmatrix}$

		\task $B = \begin{bmatrix}
				1 & 2 \\
				2 & 4
			\end{bmatrix}$
	\end{tasks}

	\vspace{.3cm}
	\textbf{Resolução} \vspace{.3cm}

	\begin{tasks}
		\task {
			Admitindo que $A^{-1} = \begin{bmatrix}
					a & b \\
					c & d
				\end{bmatrix}$ seja a inversa da matriz \textit{A}, devemos ter
			$A \cdot A^{-1} = I_2$, ou seja:
			\begin{equation*}
				\begin{bmatrix}
					1 & 3 \\
					0 & 2
				\end{bmatrix} \cdot \begin{bmatrix}
					a & b \\
					c & d
				\end{bmatrix} = \begin{bmatrix}
					1 & 0 \\
					0 & 1
				\end{bmatrix} \implies \begin{bmatrix}
					a+3c & b+3d \\
					2c   & 2d
				\end{bmatrix} = \begin{bmatrix}
					1 & 0 \\
					0 & 1
				\end{bmatrix}
			\end{equation*}
			Igualando as matrizes, encontramos:
			\systeme{
				a + 3c = 1,
				2c = 0,
				b + 3d = 0,
				2d = 1
			} $\implies $
			\systeme{
				a + 3c = 1@(1),
				c = 0@(2),
				b+3d = 0@(3),
				d=\frac{1}{2}@(4)
			}

			Substituindo (2) em (1), obtemos: $a = 1$

			Substituindo (4) em (3), obtemos: $b = -\frac{3}{2}$

			Assim, concluímos: $A^{-1} = \begin{bmatrix}
					1 & -\frac{3}{2} \\
					0 & \frac{1}{2}
				\end{bmatrix}$
		}

		\task {
			Admitindo que $B^{-1} = \begin{bmatrix}
					a & b \\
					c & d
				\end{bmatrix}$ seja a inversa da matriz \textit{B}, devemos ter
			$B \cdot B^{-1} = I_2$, ou seja:
			\begin{equation*}
				\begin{bmatrix}
					1 & 2 \\
					2 & 4
				\end{bmatrix} \cdot \begin{bmatrix}
					a & b \\
					c & d
				\end{bmatrix} = \begin{bmatrix}
					1 & 0 \\
					0 & 1
				\end{bmatrix} \implies \begin{bmatrix}
					a+2c  & b+2d  \\
					2a+4c & 2b+4d
				\end{bmatrix} = \begin{bmatrix}
					1 & 0 \\
					0 & 1
				\end{bmatrix}
			\end{equation*}

			Igualando as matrizes, encontramos:
			\systeme{
				a+2c = 1,
				2a+4c = 0,
				b+2d = 0,
				2b+4d=1
			} $\implies $
			\systeme{
				a+2c = 1@(1),
				a+2c=0@(2),
				b+2d=0@(3),
				b+2d=\frac{1}{2}@(4)
			}

			Logo, o sistema é impossível de responder, pois não existe solução e dessa forma
			não existe matriz inversa de \textit{B}.
		}
	\end{tasks}
\end{exercise}

\subsection{Exercícios Propostos}

\begin{enumerate}[label*=\protect\fbox{\arabic{enumi}}]
	\item {
	      Obtenha, se existir, a inversa de cada matriz:
	      \begin{tasks}(4)
		      \task $A = \begin{bmatrix}
				      3 & 6 \\
				      0 & 1
			      \end{bmatrix}$
		      \task $B = \begin{bmatrix}
				      3 & 5 \\
				      1 & 2
			      \end{bmatrix}$
		      \task $C = \begin{bmatrix}
				      1 & 1 \\
				      1 & 1
			      \end{bmatrix}$
		      \task $D = \begin{bmatrix}
				      0 & 2 & 0 \\
				      0 & 0 & 1 \\
				      1 & 0 & 0
			      \end{bmatrix}$
	      \end{tasks}
	      }
\end{enumerate}
