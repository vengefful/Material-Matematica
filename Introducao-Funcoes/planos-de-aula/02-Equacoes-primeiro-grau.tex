\documentclass[oneside,a4paper,12pt]{article}

\usepackage[english,brazilian]{babel}
\usepackage[utf8]{inputenc}
\usepackage[T1]{fontenc}

\usepackage[top=20mm, bottom=20mm, left=20mm, right=20mm]{geometry}
\usepackage{framed}
\usepackage{booktabs}	   		% Pacote para deixar tabelas mais bonitas.
\usepackage{color}				% Pacote de Cores
\usepackage{hyperref}			% Pacotes para Hiperlinks
\usepackage{graphicx}			% Pacote de imagens

\definecolor{shadecolor}{rgb}{0.8,0.8,0.8}

\newcommand{\universidade}{Escola Estadual Professor Lima Castro}
\newcommand{\professores}{Fernando Jorge}
\newcommand{\disciplina}{Matemática}
\newcommand{\tema}{Equações do Primeiro Grau com uma incógnita}
\newcommand{\turma}{1ºs anos}
\newcommand{\data}{04/07/2023 - 06/07/2023}


\begin{document}

\pagestyle{empty}

\begin{center}

	\universidade
	\par
	\vspace{10pt}
	\LARGE \textbf{Plano de Aula Semanal}

\end{center}

\vspace{10pt}

\begin{tabular}{ |l|p{12cm}| }

	\hline
	\multicolumn{2}{|l|}{\textbf{Dados de Identificação}} \\
	\hline
	Professores: & \professores                           \\
	\hline
	Disciplina:  & \disciplina                            \\
	\hline
	Tema:        & \tema                                  \\
	\hline
	Turma:       & \turma                                 \\
	\hline
	Data:        & \data                                  \\
	\hline
\end{tabular}

\begin{snugshade}
	\section{Objetivos} % a serem alcançados pelos alunos e não pelo professor. Podem ser divididos em gerais e específicos.
\end{snugshade}

\subsection{Geral} % projeta resultado geral relativo a execução de conteúdos e procedimentos.

Revisão de Equações do Primeiro Grau com uma incógnita.

\subsection{Específicos} % especificam resultados esperados observáveis (geralmente de 3 a 4).

\begin{itemize}

	\item Interpretar problemas contextualizados envolvendo equações do primeiro grau com uma incógnita;
	\item Resolver equações de primeiro com uma incógnita utilizando o algoritmo proposto;

\end{itemize}

\begin{snugshade}
	\section{Conteúdos} % conteúdos programados para a aula organizados em tópicos (de 4 a 8).
\end{snugshade}

\begin{itemize}

	\item Equações de primeiro grau com uma incógnita.

\end{itemize}

\begin{snugshade}
	\section{Procedimentos metodológicos} % estratégias relevantes adotadas para alcançar os objetivos.
\end{snugshade}

Apresentação expositiva e dialogada. Problematização de Estudos de Casos.

\begin{snugshade}
	\section{Recursos didáticos} % quadro, giz, retro-projetor, filme, música, quadrinhos, etc.
\end{snugshade}

\begin{itemize}

	\item Pincel e quadro.

\end{itemize}

\begin{snugshade}
	\section{Avaliação} % pode ser realizada com diferentes propósitos (diagnóstica, formativa e somativa). Interessante explicitar a atividade avaliativa e os critérios de correção.
\end{snugshade}

Alunos devem demonstrar compreensão e resolver problemas contextualizados utilizando o algoritmo de resolução de
equações do primeiro grau com uma incógnita.

%\cleardoublepage

% Referências bibliográficas


\begin{thebibliography}{}

	\bibitem{paiva2013matematica}
	PAIVA, Manoel Rodrigues.
	\newblock \textbf{Matemática Paiva 1,}
	\newblock Editora: Moderna Plus, 2. ed., São Paulo, 2010.

\end{thebibliography}

\end{document}
