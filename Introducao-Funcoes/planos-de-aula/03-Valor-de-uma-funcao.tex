\documentclass[oneside,a4paper,12pt]{article}

\usepackage[english,brazilian]{babel}
\usepackage[utf8]{inputenc}
\usepackage[T1]{fontenc}

\usepackage[top=20mm, bottom=20mm, left=20mm, right=20mm]{geometry}
\usepackage{framed}
\usepackage{booktabs}	   		% Pacote para deixar tabelas mais bonitas.
\usepackage{color}				% Pacote de Cores
\usepackage{hyperref}			% Pacotes para Hiperlinks
\usepackage{graphicx}			% Pacote de imagens

\definecolor{shadecolor}{rgb}{0.8,0.8,0.8}

\newcommand{\universidade}{Escola Estadual Professor Lima Castro}
\newcommand{\professores}{Fernando Jorge}
\newcommand{\disciplina}{Matemática}
\newcommand{\tema}{Função Afim}
\newcommand{\turma}{1ºs anos}
\newcommand{\data}{25/07/2023 - 01/08/2023}


\begin{document}

  \pagestyle{empty}

	\begin{center}

	  \universidade
	  \par
	  \vspace{10pt}
	  \LARGE \textbf{Plano de Aula Semanal}

	\end{center}

  \vspace{10pt}

	\begin{tabular}{ |l|p{12cm}| }

	  \hline
	  \multicolumn{2}{|l|}{\textbf{Dados de Identificação}} \\
	  \hline
	  Professores:         &    \professores           \\
	  \hline
	  Disciplina:        &    \disciplina          \\
	  \hline
	  Tema:              &    \tema                \\
	  \hline
	  Turma:             &    \turma               \\
	  \hline
	  Data:              &    \data                \\
	  \hline

	\end{tabular}

  \begin{snugshade}
  \section{Objetivos} % a serem alcançados pelos alunos e não pelo professor. Podem ser divididos em gerais e específicos.
  \end{snugshade}

  \subsection{Geral} % projeta resultado geral relativo a execução de conteúdos e procedimentos.

  Trabalhar o conceito do valor de uma função usando funcṍes afins.

  \subsection{Específicos} % especificam resultados esperados observáveis (geralmente de 3 a 4).

    \begin{itemize}

      \item Aprender os conceitos do valor de uma função por meio da linguagem
          algébrica;
      \item Associar o conceito de valor de uma função com problemas do cotidiano
          (contextualizado);
      \item Praticar operações básicas com números inteiros através do valor de uma função afim.

    \end{itemize}

  \begin{snugshade}
  \section{Conteúdos} % conteúdos programados para a aula organizados em tópicos (de 4 a 8).
  \end{snugshade}

    \begin{itemize}

      \item Função Afim: Valor de uma função;

    \end{itemize}

  \begin{snugshade}
  \section{Procedimentos metodológicos} % estratégias relevantes adotadas para alcançar os objetivos.
  \end{snugshade}

	Apresentação expositiva e dialogada. Problematização de Estudos de Casos e associação com
    a linguagem algébrica na matemática.

  \begin{snugshade}
  \section{Recursos didáticos} % quadro, giz, retro-projetor, filme, música, quadrinhos, etc.
  \end{snugshade}

    \begin{itemize}

	  \item projetor multimídia, pincel e quadro.

    \end{itemize}

  \begin{snugshade}
  \section{Avaliação} % pode ser realizada com diferentes propósitos (diagnóstica, formativa e somativa). Interessante explicitar a atividade avaliativa e os critérios de correção.
  \end{snugshade}

  Alunos devem demonstrar compreensão no conceito de valor de uma função, associando-a a elementos do cotidiano.

%\cleardoublepage

% Referências bibliográficas


\begin{thebibliography}{}

\bibitem{paiva2013matematica}
PAIVA, Manoel Rodrigues.
\newblock \textbf{Matemática Paiva 1,}
\newblock Editora: Moderna Plus, 2. ed., São Paulo, 2010.

\end{thebibliography}

\end{document}
