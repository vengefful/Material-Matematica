%%Template made by Uday Khankhoje for examinations using the exam template
%%Refer to the documentation http://www-math.mit.edu/~psh/exam/examdoc.pdf
%%for lot more bells and whistles to the standard template shown below
\documentclass[a4paper,11pt,addpoints]{exam}
\usepackage[left=1.5cm,right=1.5cm,top=1.5cm,bottom=2cm]{geometry}
%\usepackage{mathrsfs}
\usepackage{graphicx,color}
\usepackage[x11names]{xcolor}
\usepackage{venndiagram}
\usepackage{epic,eepic}
%\usepackage{mathpazo}
\usepackage{url}
\usepackage{tasks} % cria lista curta
\usepackage{multicol}
\usepackage{amsmath, amsthm, amssymb}
\pointsinmargin
\boxedpoints
\renewcommand*\half{.5}
\usepackage{setspace}
\DeclareMathOperator{\vecc}{vec}
%\renewcommand{\vec}[1]{\ensuremath{\mathbf{#1}}}

\global\vbadness=1616

\begin{document}
\noindent
%%PART 1 of header
\begin{center}
	\vspace*{-3em}
	\def\arraystretch{2.0}
	\begin{tabular}{|p{0.7\linewidth}|p{0.2\linewidth}|}
		\hline
		\textbf{Avaliação de Matemática - Segundo Bimestre}                                                           & Pontos Obtidos $\downarrow$ \\
		\hline
		Data:\hspace{3cm}  Total de questões \textbf{\numquestions} \hspace{1cm} Total de pontos: \textbf{\numpoints} &                             \\
		\hline
		\multicolumn{2}{|l|}{Tuma: \hspace{0.3\linewidth} Nome: \hspace{0.3\linewidth} Duração: 1 hr}                                               \\
		\hline
	\end{tabular}
\end{center}
%%PART 2 of header, if you have too many questions, this may be a problem
%%if so, use \multirowgradetable{n}[questions], where n is the number of rows you want
%%or, switch to \gradetable[h][pages] instead,
\begin{center}
	\gradetable[h][questions]
\end{center}
%%PART 3 of header
\textbf{Instruções
	\begin{enumerate}
		\item Explique todas as questões claramente.
		\item Necessário todos os cálculos.
	\end{enumerate}
}
%%toggle comment on next line to show/hide the answers
% \printanswers
%%Now the actual paper!
\begin{questions}

	\question[1]

	Resolva as equações abaixo:

	\begin{tasks}(2)
		\task $4x = -2x + 24$
		\task $2x + 5 = x + 5$
		\task $6x + 5 - 3x = 14$
		\task $x + 2x + 3 - 5x = 4x - 9$
	\end{tasks}

	\question[1]

	Resolva as equações abaixo:

	\begin{tasks}(2)
		\task $5x - 1 = 3(x - 1)$
		\task $7(x-4) = 2x - 3$
		\task $5x - 3(x + 2) = 16$
		\task $3(2x + 3) - 4(x - 1) = 3$
	\end{tasks}

	\question[1]

	O dobro de um número, menos 10, é igual a sua metade, mais 50. Qual é esse
	número?

	\question[1]

	Em uma partida de basquete, todos os 86 pontos de um time foram marcados por
	apenas três jogadores: Adão, Aldo e Amauri. Se Adão marcou 10 pontos a mais
	que Amauri e 9 pontos a menos que Aldo, quantos pontos cada jogador marcou?

	\question[1]

	Três irmãos herdaram 81 vacas. O do meio recebeu o dobro das vacas que
	recebeu o mais novo. O mais velho recebeu o triplo das vacas que recebeu
	o do meio. Quantas vacas recebeu cada um?


\end{questions}
\end{document}
