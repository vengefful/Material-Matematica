%%Template made by Uday Khankhoje for examinations using the exam template
%%Refer to the documentation http://www-math.mit.edu/~psh/exam/examdoc.pdf
%%for lot more bells and whistles to the standard template shown below
\documentclass[a4paper,11pt,addpoints]{exam}
\usepackage[left=1.5cm,right=1.5cm,top=1.5cm,bottom=2cm]{geometry}
%\usepackage{mathrsfs}
\usepackage{graphicx,color}
\usepackage[x11names]{xcolor}
\usepackage{venndiagram}
\usepackage{epic,eepic}
%\usepackage{mathpazo}
\usepackage{url}
\usepackage{tasks} % cria lista curta
\usepackage{multicol}
\usepackage{amsmath, amsthm, amssymb}
\pointsinmargin
\boxedpoints
\renewcommand*\half{.5}
\usepackage{setspace}
\DeclareMathOperator{\vecc}{vec}
%\renewcommand{\vec}[1]{\ensuremath{\mathbf{#1}}}

\global\vbadness=1616

\begin{document}
\noindent
%%PART 1 of header
\begin{center}
	\vspace*{-3em}
	\def\arraystretch{2.0}
	\begin{tabular}{|p{0.7\linewidth}|p{0.2\linewidth}|}
		\hline
		\textbf{Avaliação de Matemática - Segundo Bimestre}                                                           & Pontos Obtidos $\downarrow$ \\
		\hline
		Data:\hspace{3cm}  Total de questões \textbf{\numquestions} \hspace{1cm} Total de pontos: \textbf{\numpoints} &                             \\
		\hline
		\multicolumn{2}{|l|}{Tuma: \hspace{0.3\linewidth} Nome: \hspace{0.3\linewidth} Duração: 1 hr}                                               \\
		\hline
	\end{tabular}
\end{center}
%%PART 2 of header, if you have too many questions, this may be a problem
%%if so, use \multirowgradetable{n}[questions], where n is the number of rows you want
%%or, switch to \gradetable[h][pages] instead,
\begin{center}
	\gradetable[h][questions]
\end{center}
%%PART 3 of header
\textbf{Instruções
	\begin{enumerate}
		\item Explique todas as questões claramente.
		\item Necessário todos os cálculos.
	\end{enumerate}
}
%%toggle comment on next line to show/hide the answers
% \printanswers
%%Now the actual paper!
\begin{questions}

	\question[2]

    A média das notas na prova de Matemática de uma turma com 30 alunos foi de
    70 alunos. Nenhum dos alunos obteve nota inferior a 60 pontos. O número
    máximo de alunos que podem ter obtido nota igual a 90 pontos é:

    \begin{tasks}(4)
        \task 16
        \task 13
        \task 23
        \task 10
    \end{tasks}

	\question[2]

    Um casal tem filhos e filhas. Cada filho tem o número de irmãos igual ao
    número de irmãs. Cada filha tem o número de irmãos igual ao dobro do número
    de irmãs. Qual é o totalde filhos e filhas do casal?

    \begin{tasks}(5)
        \task 3
        \task 4
        \task 5
        \task 6
        \task 7
    \end{tasks}

	\question[2]

    Um determinado medicamento deve ser administrado a um doente três vezes ao dia,
    em doses de 5ml cada vez, durante 10 dias. Se cada frasco contém 100 $cm^3$ do
    medicamento, o número de frascos necessários é:

    \begin{tasks}(5)
        \task 2.5
        \task 1
        \task 1.5
        \task 2
        \task 3
    \end{tasks}

    \question[2]

    Duas empreiteras farão conjuntamente a pavimentação de uma estrada, cada uma
    trabalhando a partir de uma das extremidades. Se uma delas pavimentar
    $\dfrac{2}{5}$ da estrada e a outra os 81 km restantes, a extensão dessa
    estrada é de:

    \begin{tasks}(5)
        \task 125 km
        \task 135 km
        \task 142 km
        \task 145 km
        \task 160 km
    \end{tasks}


\end{questions}
\end{document}
